\chapter{深度優先搜索}


\section{Palindrome Partitioning} %%%%%%%%%%%%%%%%%%%%%%%%%%%%%%
\label{sec:palindrome-partitioning}


\subsubsection{描述}
Given a string s, partition s such that every substring of the partition is a palindrome.

Return all possible palindrome partitioning of s.

For example, given \code{s = "aab"},
Return
\begin{Code}
  [
    ["aa","b"],
    ["a","a","b"]
  ]
\end{Code}


\subsubsection{分析}
在每一步都可以判斷中間結果是否為合法結果,用回溯法。

一個長度為n的字符串,有$n-1$個地方可以砍斷,每個地方可斷可不斷,因此複雜度為$O(2^{n-1})$


\subsubsection{深搜1}
\begin{Code}
//LeetCode, Palindrome Partitioning
// 時間複雜度O(2^n),空間複雜度O(n)
class Solution {
public:
    vector<vector<string>> partition(string s) {
        vector<vector<string>> result;
        vector<string> path;  // 一個partition方案
        dfs(s, path, result, 0, 1);
        return result;
    }

    // prev 表示前一個隔板, start 表示當前隔板
    void dfs(string &s, vector<string>& path,
            vector<vector<string>> &result, size_t prev, size_t start) {
        if (start == s.size()) { // 最後一個隔板
            if (isPalindrome(s, prev, start - 1)) { // 必須使用
                path.push_back(s.substr(prev, start - prev));
                result.push_back(path);
                path.pop_back();
            }
            return;
        }
        // 不斷開
        dfs(s, path, result, prev, start + 1);
        // 如果[prev, start-1] 是迴文,則可以斷開,也可以不斷開(上一行已經做了)
        if (isPalindrome(s, prev, start - 1)) {
            // 斷開
            path.push_back(s.substr(prev, start - prev));
            dfs(s, path, result, start, start + 1);
            path.pop_back();
        }
    }

    bool isPalindrome(const string &s, int start, int end) {
        while (start < end && s[start] == s[end]) {
            ++start;
            --end;
        }
        return start >= end;
    }
};
\end{Code}

\subsubsection{深搜2}
另一種寫法,更加簡潔。這種寫法也在 Combination Sum, Combination Sum II 中出現過。
\begin{Code}
//LeetCode, Palindrome Partitioning
// 時間複雜度O(2^n),空間複雜度O(n)
class Solution {
public:
    vector<vector<string>> partition(string s) {
        vector<vector<string>> result;
        vector<string> path;  // 一個partition方案
        DFS(s, path, result, 0);
        return result;
    }
    // 搜索必須以s[start]開頭的partition方案
    void DFS(string &s, vector<string>& path,
            vector<vector<string>> &result, int start) {
        if (start == s.size()) {
            result.push_back(path);
            return;
        }
        for (int i = start; i < s.size(); i++) {
            if (isPalindrome(s, start, i)) { // 從i位置砍一刀
                path.push_back(s.substr(start, i - start + 1));
                DFS(s, path, result, i + 1);  // 繼續往下砍
                path.pop_back(); // 撤銷上上行
            }
        }
    }
    bool isPalindrome(const string &s, int start, int end) {
        while (start < end && s[start] == s[end]) {
            ++start;
            --end;
        }
        return start >= end;
    }
};
\end{Code}


\subsubsection{動規}
\begin{Code}
// LeetCode, Palindrome Partitioning
// 動規,時間複雜度O(n^2),空間複雜度O(1)
class Solution {
public:
    vector<vector<string> > partition(string s) {
        const int n = s.size();
        bool p[n][n]; // whether s[i,j] is palindrome
        fill_n(&p[0][0], n * n, false);
        for (int i = n - 1; i >= 0; --i)
            for (int j = i; j < n; ++j)
                p[i][j] = s[i] == s[j] && ((j - i < 2) || p[i + 1][j - 1]);

        vector<vector<string> > sub_palins[n]; // sub palindromes of s[0,i]
        for (int i = n - 1; i >= 0; --i) {
            for (int j = i; j < n; ++j)
                if (p[i][j]) {
                    const string palindrome = s.substr(i, j - i + 1);
                    if (j + 1 < n) {
                        for (auto v : sub_palins[j + 1]) {
                            v.insert(v.begin(), palindrome);
                            sub_palins[i].push_back(v);
                        }
                    } else {
                        sub_palins[i].push_back(vector<string> { palindrome });
                    }
                }
        }
        return sub_palins[0];
    }
};
\end{Code}


\subsubsection{相關題目}

\begindot
\item Palindrome Partitioning II,見 \S \ref{sec:palindrome-partitioning-ii}
\myenddot


\section{Unique Paths} %%%%%%%%%%%%%%%%%%%%%%%%%%%%%%
\label{sec:unique-paths}


\subsubsection{描述}
A robot is located at the top-left corner of a $m \times n$ grid (marked 'Start' in the diagram below).

The robot can only move either down or right at any point in time. The robot is trying to reach the bottom-right corner of the grid (marked 'Finish' in the diagram below).

How many possible unique paths are there?

\begin{center}
\includegraphics[width=200pt]{robot-maze.png}\\
\figcaption{Above is a $3 \times 7$ grid. How many possible unique paths are there?}\label{fig:unique-paths}
\end{center}

\textbf{Note}: $m$ and $n$ will be at most 100.


\subsection{深搜}
深搜,小集合可以過,大集合會超時

\subsubsection{代碼}
\begin{Code}
// LeetCode, Unique Paths
// 深搜,小集合可以過,大集合會超時
// 時間複雜度O(n^4),空間複雜度O(n)
class Solution {
public:
    int uniquePaths(int m, int n) {
        if (m < 1 || n < 1) return 0; // 終止條件

        if (m == 1 && n == 1) return 1; // 收斂條件

        return uniquePaths(m - 1, n) + uniquePaths(m, n - 1);
    }
};
\end{Code}


\subsection{備忘錄法}
給前面的深搜,加個緩存,就可以過大集合了。即備忘錄法。

\subsubsection{代碼}
\begin{Code}
// LeetCode, Unique Paths
// 深搜 + 緩存,即備忘錄法
// 時間複雜度O(n^2),空間複雜度O(n^2)
class Solution {
public:
    int uniquePaths(int m, int n) {
        // f[x][y] 表示 從(0,0)到(x,y)的路徑條數
        f = vector<vector<int> >(m, vector<int>(n, 0));
        f[0][0] = 1;
        return dfs(m - 1, n - 1);
    }
private:
    vector<vector<int> > f;  // 緩存

    int dfs(int x, int y) {
        if (x < 0 || y < 0) return 0; // 數據非法,終止條件

        if (x == 0 && y == 0) return f[0][0]; // 回到起點,收斂條件

        if (f[x][y] > 0) {
            return f[x][y];
        } else {
            return f[x][y] = dfs(x - 1, y) +  dfs(x, y - 1);
        }
    }
};
\end{Code}


\subsection{動規}
既然可以用備忘錄法自頂向下解決,也一定可以用動規自底向上解決。

設狀態為\fn{f[i][j]},表示從起點$(1,1)$到達$(i,j)$的路線條數,則狀態轉移方程為:
\begin{Code}
f[i][j]=f[i-1][j]+f[i][j-1]
\end{Code}


\subsubsection{代碼}
\begin{Code}
// LeetCode, Unique Paths
// 動規,滾動數組
// 時間複雜度O(n^2),空間複雜度O(n)
class Solution {
public:
    int uniquePaths(int m, int n) {
        vector<int> f(n, 0);
        f[0] = 1;
        for (int i = 0; i < m; i++) {
            for (int j = 1; j < n; j++) {
                // 左邊的f[j],表示更新後的f[j],與公式中的f[i][j]對應
                // 右邊的f[j],表示老的f[j],與公式中的f[i-1][j]對應
                f[j] = f[j] + f[j - 1];
            }
        }
        return f[n - 1];
    }
};
\end{Code}


\subsection{數學公式}
一個$m$行,$n$列的矩陣,機器人從左上走到右下總共需要的步數是$m+n-2$,其中向下走的步數是$m-1$,因此問題變成了在$m+n-2$個操作中,選擇$m–1$個時間點向下走,選擇方式有多少種。即 $C_{m+n-2}^{m-1}$ 。

\subsubsection{代碼}
\begin{Code}
// LeetCode, Unique Paths
// 數學公式
class Solution {
public:
    typedef long long int64_t;
    // 求階乘, n!/(start-1)!,即 n*(n-1)...start,要求 n >= 1
    static int64_t factor(int n, int start = 1) {
        int64_t  ret = 1;
        for(int i = start; i <= n; ++i)
            ret *= i;
        return ret;
    }
    // 求組合數 C_n^k
    static int64_t combination(int n, int k) {
        // 常數優化
        if (k == 0) return 1;
        if (k == 1) return n;

        int64_t ret = factor(n, k+1);
        ret /= factor(n - k);
        return ret;
    }

    int uniquePaths(int m, int n) {
        // max 可以防止n和k差距過大,從而防止combination()溢出
        return combination(m+n-2, max(m-1, n-1));
    }
};
\end{Code}


\subsubsection{相關題目}
\begindot
\item Unique Paths II,見 \S \ref{sec:unique-paths-ii}
\item Minimum Path Sum, 見 \S \ref{sec:minimum-path-sum}
\myenddot


\section{Unique Paths II} %%%%%%%%%%%%%%%%%%%%%%%%%%%%%%
\label{sec:unique-paths-ii}


\subsubsection{描述}
Follow up for "Unique Paths":

Now consider if some obstacles are added to the grids. How many unique paths would there be?

An obstacle and empty space is marked as 1 and 0 respectively in the grid.

For example,

There is one obstacle in the middle of a $3 \times 3$ grid as illustrated below.
\begin{Code}
[
  [0,0,0],
  [0,1,0],
  [0,0,0]
]
\end{Code}

The total number of unique paths is 2.

Note: $m$ and $n$ will be at most 100.


\subsection{備忘錄法}
在上一題的基礎上改一下即可。相比動規,簡單得多。

\subsubsection{代碼}
\begin{Code}
// LeetCode, Unique Paths II
// 深搜 + 緩存,即備忘錄法
class Solution {
public:
    int uniquePathsWithObstacles(const vector<vector<int> >& obstacleGrid) {
        const int m = obstacleGrid.size();
        const int n = obstacleGrid[0].size();
        if (obstacleGrid[0][0] || obstacleGrid[m - 1][n - 1]) return 0;

        f = vector<vector<int> >(m, vector<int>(n, 0));
        f[0][0] = obstacleGrid[0][0] ? 0 : 1;
        return dfs(obstacleGrid, m - 1, n - 1);
    }
private:
    vector<vector<int> > f;  // 緩存

    // @return 從 (0, 0) 到 (x, y) 的路徑總數
    int dfs(const vector<vector<int> >& obstacleGrid,
            int x, int y) {
        if (x < 0 || y < 0) return 0; // 數據非法,終止條件

        // (x,y)是障礙
        if (obstacleGrid[x][y]) return 0;

        if (x == 0 and y == 0) return f[0][0]; // 回到起點,收斂條件

        if (f[x][y] > 0) {
            return f[x][y];
        } else {
            return f[x][y] = dfs(obstacleGrid, x - 1, y) + 
                dfs(obstacleGrid, x, y - 1);
        }
    }
};
\end{Code}


\subsection{動規}
與上一題類似,但要特別注意第一列的障礙。在上一題中,第一列全部是1,但是在這一題中不同,第一列如果某一行有障礙物,那麼後面的行全為0。


\subsubsection{代碼}
\begin{Code}
// LeetCode, Unique Paths II
// 動規,滾動數組
// 時間複雜度O(n^2),空間複雜度O(n)
class Solution {
public:
    int uniquePathsWithObstacles(vector<vector<int> > &obstacleGrid) {
        const int m = obstacleGrid.size();
        const int n = obstacleGrid[0].size();
        if (obstacleGrid[0][0] || obstacleGrid[m-1][n-1]) return 0;

        vector<int> f(n, 0);
        f[0] = obstacleGrid[0][0] ? 0 : 1;

        for (int i = 0; i < m; i++) {
            f[0] = f[0] == 0 ? 0 : (obstacleGrid[i][0] ? 0 : 1);
            for (int j = 1; j < n; j++)
                f[j] = obstacleGrid[i][j] ? 0 : (f[j] + f[j - 1]);
        }

        return f[n - 1];
    }
};
\end{Code}


\subsubsection{相關題目}
\begindot
\item Unique Paths,見 \S \ref{sec:unique-paths}
\item Minimum Path Sum, 見 \S \ref{sec:minimum-path-sum}
\myenddot


\section{N-Queens} %%%%%%%%%%%%%%%%%%%%%%%%%%%%%%
\label{sec:n-queens}


\subsubsection{描述}
The \emph{n-queens puzzle} is the problem of placing n queens on an $n \times n$ chessboard such that no two queens attack each other.

\begin{center}
\includegraphics{8-queens.png}\\
\figcaption{Eight Queens}\label{fig:8-queens}
\end{center}

Given an integer $n$, return all distinct solutions to the n-queens puzzle.

Each solution contains a distinct board configuration of the n-queens' placement, where \fn{'Q'} and \fn{'.'} both indicate a queen and an empty space respectively.

For example,
There exist two distinct solutions to the 4-queens puzzle:
\begin{Code}
[
 [".Q..",  // Solution 1
  "...Q",
  "Q...",
  "..Q."],

 ["..Q.",  // Solution 2
  "Q...",
  "...Q",
  ".Q.."]
]
\end{Code}


\subsubsection{分析}

經典的深搜題。

設置一個數組 \fn{vector<int> C(n, 0)}, \fn{C[i]} 表示第i行皇后所在的列編號,即在位置 (i, C[i]) 上放了一個皇后,這樣用一個一維數組,就能記錄整個棋盤。


\subsubsection{代碼1}
\begin{Code}
// LeetCode, N-Queens
// 深搜+剪枝
// 時間複雜度O(n!*n),空間複雜度O(n)
class Solution {
public:
    vector<vector<string> > solveNQueens(int n) {
        vector<vector<string> > result;
        vector<int> C(n, -1);  // C[i]表示第i行皇后所在的列編號
        dfs(C, result, 0);
        return result;
    }
private:
    void dfs(vector<int> &C, vector<vector<string> > &result, int row) {
        const int N = C.size();
        if (row == N) { // 終止條件,也是收斂條件,意味着找到了一個可行解
            vector<string> solution;
            for (int i = 0; i < N; ++i) {
                string s(N, '.');
                for (int j = 0; j < N; ++j) {
                    if (j == C[i]) s[j] = 'Q';
                }
                solution.push_back(s);
            }
            result.push_back(solution);
            return;
        }

        for (int j = 0; j < N; ++j) {  // 擴展狀態,一列一列的試
            const bool ok = isValid(C, row, j);
            if (!ok) continue;  // 剪枝,如果非法,繼續嘗試下一列
            // 執行擴展動作
            C[row] = j;
            dfs(C, result, row + 1);
            // 撤銷動作
            // C[row] = -1;
        }
    }
    
    /**
     * 能否在 (row, col) 位置放一個皇后.
     *
     * @param C 棋局
     * @param row 當前正在處理的行,前面的行都已經放了皇后了
     * @param col 當前列
     * @return 能否放一個皇后
     */
    bool isValid(const vector<int> &C, int row, int col) {
        for (int i = 0; i < row; ++i) {
            // 在同一列
            if (C[i] == col) return false;
            // 在同一對角線上
            if (abs(i - row) == abs(C[i] - col)) return false;
        }
        return true;
    }
};
\end{Code}


\subsubsection{代碼2}
\begin{Code}
// LeetCode, N-Queens
// 深搜+剪枝
// 時間複雜度O(n!),空間複雜度O(n)
class Solution {
public:
    vector<vector<string> > solveNQueens(int n) {
        this->columns = vector<bool>(n, false);
        this->main_diag = vector<bool>(2 * n - 1, false);
        this->anti_diag = vector<bool>(2 * n - 1, false);

        vector<vector<string> > result;
        vector<int> C(n, -1);  // C[i]表示第i行皇后所在的列編號
        dfs(C, result, 0);
        return result;
    }
private:
    // 這三個變量用於剪枝
    vector<bool> columns;  // 表示已經放置的皇后佔據了哪些列
    vector<bool> main_diag;  // 佔據了哪些主對角線
    vector<bool> anti_diag;  // 佔據了哪些副對角線

    void dfs(vector<int> &C, vector<vector<string> > &result, int row) {
        const int N = C.size();
        if (row == N) { // 終止條件,也是收斂條件,意味着找到了一個可行解
            vector<string> solution;
            for (int i = 0; i < N; ++i) {
                string s(N, '.');
                for (int j = 0; j < N; ++j) {
                    if (j == C[i]) s[j] = 'Q';
                }
                solution.push_back(s);
            }
            result.push_back(solution);
            return;
        }

        for (int j = 0; j < N; ++j) {  // 擴展狀態,一列一列的試
            const bool ok = !columns[j] && !main_diag[row - j + N - 1]  &&
                    !anti_diag[row + j];
            if (!ok) continue;  // 剪枝,如果非法,繼續嘗試下一列
            // 執行擴展動作
            C[row] = j;
            columns[j] = main_diag[row - j + N - 1] = anti_diag[row + j] = true;
            dfs(C, result, row + 1);
            // 撤銷動作
            // C[row] = -1;
            columns[j] = main_diag[row - j + N - 1] = anti_diag[row + j] = false;
        }
    }
};
\end{Code}


\subsubsection{相關題目}
\begindot
\item N-Queens II,見 \S \ref{sec:n-queens-ii}
\myenddot


\section{N-Queens II} %%%%%%%%%%%%%%%%%%%%%%%%%%%%%%
\label{sec:n-queens-ii}


\subsubsection{描述}
Follow up for N-Queens problem.

Now, instead outputting board configurations, return the total number of distinct solutions.


\subsubsection{分析}
只需要輸出解的個數,不需要輸出所有解,代碼要比上一題簡化很多。設一個全局計數器,每找到一個解就增1。


\subsubsection{代碼1}
\begin{Code}
// LeetCode, N-Queens II
// 深搜+剪枝
// 時間複雜度O(n!*n),空間複雜度O(n)
class Solution {
public:
    int totalNQueens(int n) {
        this->count = 0;

        vector<int> C(n, 0);  // C[i]表示第i行皇后所在的列編號
        dfs(C, 0);
        return this->count;
    }
private:
    int count; // 解的個數

    void dfs(vector<int> &C, int row) {
        const int N = C.size();
        if (row == N) { // 終止條件,也是收斂條件,意味着找到了一個可行解
            ++this->count;
            return;
        }

        for (int j = 0; j < N; ++j) {  // 擴展狀態,一列一列的試
            const bool ok = isValid(C, row, j);
            if (!ok) continue;  // 剪枝:如果合法,繼續遞歸
            // 執行擴展動作
            C[row] = j;
            dfs(C, row + 1);
            // 撤銷動作
            // C[row] = -1;
        }
    }
    /**
     * 能否在 (row, col) 位置放一個皇后.
     *
     * @param C 棋局
     * @param row 當前正在處理的行,前面的行都已經放了皇后了
     * @param col 當前列
     * @return 能否放一個皇后
     */
    bool isValid(const vector<int> &C, int row, int col) {
        for (int i = 0; i < row; ++i) {
            // 在同一列
            if (C[i] == col) return false;
            // 在同一對角線上
            if (abs(i - row) == abs(C[i] - col)) return false;
        }
        return true;
    }
};
\end{Code}


\subsubsection{代碼2}
\begin{Code}
// LeetCode, N-Queens II
// 深搜+剪枝
// 時間複雜度O(n!),空間複雜度O(n)
class Solution {
public:
    int totalNQueens(int n) {
        this->count = 0;
        this->columns = vector<bool>(n, false);
        this->main_diag = vector<bool>(2 * n - 1, false);
        this->anti_diag = vector<bool>(2 * n - 1, false);

        vector<int> C(n, 0);  // C[i]表示第i行皇后所在的列編號
        dfs(C, 0);
        return this->count;
    }
private:
    int count; // 解的個數
    // 這三個變量用於剪枝
    vector<bool> columns;  // 表示已經放置的皇后佔據了哪些列
    vector<bool> main_diag;  // 佔據了哪些主對角線
    vector<bool> anti_diag;  // 佔據了哪些副對角線

    void dfs(vector<int> &C, int row) {
        const int N = C.size();
        if (row == N) { // 終止條件,也是收斂條件,意味着找到了一個可行解
            ++this->count;
            return;
        }

        for (int j = 0; j < N; ++j) {  // 擴展狀態,一列一列的試
            const bool ok = !columns[j] &&
                    !main_diag[row - j + N] &&
                    !anti_diag[row + j];
            if (!ok) continue;  // 剪枝:如果合法,繼續遞歸
            // 執行擴展動作
            C[row] = j;
            columns[j] = main_diag[row - j + N] =
                    anti_diag[row + j] = true;
            dfs(C, row + 1);
            // 撤銷動作
            // C[row] = -1;
            columns[j] = main_diag[row - j + N] =
                    anti_diag[row + j] = false;
        }
    }
};
\end{Code}


\subsubsection{相關題目}
\begindot
\item N-Queens,見 \S \ref{sec:n-queens}
\myenddot


\section{Restore IP Addresses} %%%%%%%%%%%%%%%%%%%%%%%%%%%%%%
\label{sec:restore-ip-addresses}


\subsubsection{描述}
Given a string containing only digits, restore it by returning all possible valid IP address combinations.

For example:
Given \code{"25525511135"},

return \code{\["255.255.11.135", "255.255.111.35"\]}. (Order does not matter)


\subsubsection{分析}
必須要走到底部才能判斷解是否合法,深搜。


\subsubsection{代碼}
\begin{Code}
// LeetCode, Restore IP Addresses
// 時間複雜度O(n^4),空間複雜度O(n)
class Solution {
public:
    vector<string> restoreIpAddresses(const string& s) {
        vector<string> result;
        vector<string> ip; // 存放中間結果
        dfs(s, ip, result, 0);
        return result;
    }

    /**
     * @brief 解析字符串
     * @param[in] s 字符串,輸入數據
     * @param[out] ip 存放中間結果
     * @param[out] result 存放所有可能的IP地址
     * @param[in] start 當前正在處理的 index
     * @return 無
     */
    void dfs(string s, vector<string>& ip, vector<string> &result,
            size_t start) {
        if (ip.size() == 4 && start == s.size()) {  // 找到一個合法解
            result.push_back(ip[0] + '.' + ip[1] + '.' + ip[2] + '.' + ip[3]);
            return;
        }

        if (s.size() - start > (4 - ip.size()) * 3)
            return;  // 剪枝
        if (s.size() - start < (4 - ip.size()))
            return;  // 剪枝

        int num = 0;
        for (size_t i = start; i < start + 3; i++) {
            num = num * 10 + (s[i] - '0');

            if (num < 0 || num > 255) continue;  // 剪枝
            
            ip.push_back(s.substr(start, i - start + 1));
            dfs(s, ip, result, i + 1);
            ip.pop_back();
            
            if (num == 0) break;  // 不允許前綴0,但允許單個0
        }
    }
};
\end{Code}


\subsubsection{相關題目}
\begindot
\item 無
\myenddot


\section{Combination Sum} %%%%%%%%%%%%%%%%%%%%%%%%%%%%%%
\label{sec:combination-sum}


\subsubsection{描述}
Given a set of candidate numbers ($C$) and a target number ($T$), find all unique combinations in $C$ where the candidate numbers sums to $T$.

The same repeated number may be chosen from $C$ \emph{unlimited} number of times.

Note:
\begindot
\item All numbers (including target) will be positive integers.
\item Elements in a combination ($a_1, a_2, ..., a_k$) must be in non-descending order. (ie, $a_1 \leq a_2 \leq ... \leq a_k$).
\item The solution set must not contain duplicate combinations.
\myenddot

For example, given candidate set \fn{2,3,6,7} and target \fn{7}, 
A solution set is: 
\begin{Code}
[7] 
[2, 2, 3] 
\end{Code}


\subsubsection{分析}
無


\subsubsection{代碼}
\begin{Code}
// LeetCode, Combination Sum
// 時間複雜度O(n!),空間複雜度O(n)
class Solution {
public:
    vector<vector<int> > combinationSum(vector<int> &nums, int target) {
        sort(nums.begin(), nums.end());
        vector<vector<int> > result; // 最終結果
        vector<int> path; // 中間結果
        dfs(nums, path, result, target, 0);
        return result;
    }

private:
    void dfs(vector<int>& nums, vector<int>& path, vector<vector<int> > &result,
            int gap, int start) {
        if (gap == 0) {  // 找到一個合法解
            result.push_back(path);
            return;
        }
        for (size_t i = start; i < nums.size(); i++) { // 擴展狀態
            if (gap < nums[i]) return; // 剪枝

            path.push_back(nums[i]); // 執行擴展動作
            dfs(nums, path, result, gap - nums[i], i);
            path.pop_back();  // 撤銷動作
        }
    }
};
\end{Code}


\subsubsection{相關題目}
\begindot
\item Combination Sum II ,見 \S \ref{sec:combination-sum-ii}
\myenddot


\section{Combination Sum II} %%%%%%%%%%%%%%%%%%%%%%%%%%%%%%
\label{sec:combination-sum-ii}


\subsubsection{描述}
Given a collection of candidate numbers ($C$) and a target number ($T$), find all unique combinations in $C$ where the candidate numbers sums to $T$.

Each number in $C$ may only be used \emph{once} in the combination.

Note:
\begindot
\item All numbers (including target) will be positive integers.
\item Elements in a combination ($a_1, a_2, ..., a_k$) must be in non-descending order. (ie, $a_1 > a_2 > ... > a_k$).
\item The solution set must not contain duplicate combinations.
\myenddot

For example, given candidate set \fn{10,1,2,7,6,1,5} and target \fn{8}, 
A solution set is: 
\begin{Code}
[1, 7] 
[1, 2, 5] 
[2, 6] 
[1, 1, 6]
\end{Code}


\subsubsection{分析}
無


\subsubsection{代碼}
\begin{Code}
// LeetCode, Combination Sum II
// 時間複雜度O(n!),空間複雜度O(n)
class Solution {
public:
    vector<vector<int> > combinationSum2(vector<int> &nums, int target) {
        sort(nums.begin(), nums.end()); // 跟第 50 行配合,
                                        // 確保每個元素最多隻用一次
        vector<vector<int> > result;
        vector<int> path;
        dfs(nums, path, result, target, 0);
        return result;
    }
private:
    // 使用nums[start, nums.size())之間的元素,能找到的所有可行解
    static void dfs(const vector<int> &nums, vector<int> &path, 
            vector<vector<int> > &result, int gap, int start) {
        if (gap == 0) {  //  找到一個合法解
            result.push_back(path);
            return;
        }

        int previous = -1;
        for (size_t i = start; i < nums.size(); i++) {
            // 如果上一輪循環已經使用了nums[i],則本次循環就不能再選nums[i],
            // 確保nums[i]最多隻用一次
            if (previous == nums[i]) continue;

            if (gap < nums[i]) return;  // 剪枝

            previous = nums[i];

            path.push_back(nums[i]);
            dfs(nums, path, result, gap - nums[i], i + 1);
            path.pop_back();  // 恢復環境
        }
    }
};
\end{Code}


\subsubsection{相關題目}
\begindot
\item Combination Sum ,見 \S \ref{sec:combination-sum}
\myenddot


\section{Generate Parentheses } %%%%%%%%%%%%%%%%%%%%%%%%%%%%%%
\label{sec:generate-parentheses}


\subsubsection{描述}
Given $n$ pairs of parentheses, write a function to generate all combinations of well-formed parentheses.

For example, given $n = 3$, a solution set is:
\begin{Code}
"((()))", "(()())", "(())()", "()(())", "()()()"
\end{Code}

\subsubsection{分析}
小括號串是一個遞歸結構,跟單鏈表、二叉樹等遞歸結構一樣,首先想到用遞歸。

一步步構造字符串。當左括號出現次數$<n$時,就可以放置新的左括號。當右括號出現次數小於左括號出現次數時,就可以放置新的右括號。


\subsubsection{代碼1}
\begin{Code}
// LeetCode, Generate Parentheses
// 時間複雜度O(TODO),空間複雜度O(n)
class Solution {
public:
    vector<string> generateParenthesis(int n) {
        vector<string> result;
        string path;
        if (n > 0) generate(n, path, result, 0, 0);
        return result;
    }
    // l 表示 ( 出現的次數, r 表示 ) 出現的次數
    void generate(int n, string& path, vector<string> &result, int l, int r) {
        if (l == n) {
            string s(path);
            result.push_back(s.append(n - r, ')'));
            return;
        }
        
        path.push_back('(');
        generate(n, path, result, l + 1, r);
        path.pop_back();

        if (l > r) {
            path.push_back(')');
            generate(n, path, result, l, r + 1);
            path.pop_back();
        }
    }
};
\end{Code}


\subsubsection{代碼2}
另一種遞歸寫法,更加簡潔。
\begin{Code}
// LeetCode, Generate Parentheses
// @author 連城 (http://weibo.com/lianchengzju)
class Solution {
public:
    vector<string> generateParenthesis (int n) {
        if (n == 0) return vector<string>(1, "");
        if (n == 1) return vector<string> (1, "()");
        vector<string> result;

        for (int i = 0; i < n; ++i)
            for (auto inner : generateParenthesis (i))
                for (auto outer : generateParenthesis (n - 1 - i))
                    result.push_back ("(" + inner + ")" + outer);

        return result;
    }
};
\end{Code}


\subsubsection{相關題目}
\begindot
\item Valid Parentheses, 見 \S \ref{sec:valid-parentheses}
\item Longest Valid Parentheses, 見 \S \ref{sec:longest-valid-parentheses}
\myenddot


\section{Sudoku Solver} %%%%%%%%%%%%%%%%%%%%%%%%%%%%%%
\label{sec:sudoku-solver}


\subsubsection{描述}
Write a program to solve a Sudoku puzzle by filling the empty cells.

Empty cells are indicated by the character \fn{'.'}.

You may assume that there will be only one unique solution.

\begin{center}
\includegraphics[width=150pt]{sudoku.png}\\
\figcaption{A sudoku puzzle...}\label{fig:sudoku}
\end{center}

\begin{center}
\includegraphics[width=150pt]{sudoku-solution.png}\\
\figcaption{...and its solution numbers marked in red}\label{fig:sudoku-solution}
\end{center}


\subsubsection{分析}
無。


\subsubsection{代碼}
\begin{Code}
// LeetCode, Sudoku Solver
// 時間複雜度O(9^4),空間複雜度O(1)
class Solution {
public:
    bool solveSudoku(vector<vector<char> > &board) {
        for (int i = 0; i < 9; ++i)
            for (int j = 0; j < 9; ++j) {
                if (board[i][j] == '.') {
                    for (int k = 0; k < 9; ++k) {
                        board[i][j] = '1' + k;
                        if (isValid(board, i, j) && solveSudoku(board))
                            return true;
                        board[i][j] = '.';
                    }
                    return false;
                }
            }
        return true;
    }
private:
    // 檢查 (x, y) 是否合法
    bool isValid(const vector<vector<char> > &board, int x, int y) {
        int i, j;
        for (i = 0; i < 9; i++) // 檢查 y 列
            if (i != x && board[i][y] == board[x][y])
                return false;
        for (j = 0; j < 9; j++) // 檢查 x 行
            if (j != y && board[x][j] == board[x][y])
                return false;
        for (i = 3 * (x / 3); i < 3 * (x / 3 + 1); i++)
            for (j = 3 * (y / 3); j < 3 * (y / 3 + 1); j++)
                if ((i != x || j != y) && board[i][j] == board[x][y])
                    return false;
        return true;
    }
};
\end{Code}


\subsubsection{相關題目}
\begindot
\item Valid Sudoku, 見 \S \ref{sec:valid-sudoku}
\myenddot


\section{Word Search} %%%%%%%%%%%%%%%%%%%%%%%%%%%%%%
\label{sec:word-search}


\subsubsection{描述}
Given a 2D board and a word, find if the word exists in the grid.

The word can be constructed from letters of sequentially adjacent cell, where \fn{"adjacent"} cells are those horizontally or vertically neighbouring. The same letter cell may not be used more than once.

For example,
Given board =
\begin{Code}
[
  ["ABCE"],
  ["SFCS"],
  ["ADEE"]
]
\end{Code}
word = \fn{"ABCCED"}, -> returns \fn{true},\\
word = \fn{"SEE"}, -> returns \fn{true},\\
word = \fn{"ABCB"}, -> returns \fn{false}.


\subsubsection{分析}
無。


\subsubsection{代碼}
\begin{Code}
// LeetCode, Word Search
// 深搜,遞歸
// 時間複雜度O(n^2*m^2),空間複雜度O(n^2)
class Solution {
public:
    bool exist(const vector<vector<char> > &board, const string& word) {
        const int m = board.size();
        const int n = board[0].size();
        vector<vector<bool> > visited(m, vector<bool>(n, false));
        for (int i = 0; i < m; ++i)
            for (int j = 0; j < n; ++j)
                if (dfs(board, word, 0, i, j, visited))
                    return true;
        return false;
    }
private:
    static bool dfs(const vector<vector<char> > &board, const string &word,
            int index, int x, int y, vector<vector<bool> > &visited) {
        if (index == word.size())
            return true; // 收斂條件

        if (x < 0 || y < 0 || x >= board.size() || y >= board[0].size())
            return false;  // 越界,終止條件

        if (visited[x][y]) return false; // 已經訪問過,剪枝

        if (board[x][y] != word[index]) return false; // 不相等,剪枝

        visited[x][y] = true;
        bool ret = dfs(board, word, index + 1, x - 1, y, visited) || // 上
                dfs(board, word, index + 1, x + 1, y, visited)    || // 下
                dfs(board, word, index + 1, x, y - 1, visited)    || // 左
                dfs(board, word, index + 1, x, y + 1, visited);      // 右
        visited[x][y] = false;
        return ret;
    }
};
\end{Code}


\subsubsection{相關題目}
\begindot
\item 無
\myenddot

\section{Optimal Account Balancing} %%%%%%%%%%%%%%%%%%%%%%%%%%%%%%
\label{sec:optimal-account-balancing}


\subsubsection{描述}
A group of friends went on holiday and sometimes lent each other money.
For example, Alice paid for Bill's lunch for \$10.
Then later Chris gave Alice \$5 for a taxi ride.
We can model each transaction as a tuple (x, y, z) which means person x gave person y \$z.
Assuming Alice, Bill, and Chris are person 0, 1, and 2 respectively (0, 1, 2 are the person's ID),
the transactions can be represented as [[0, 1, 10], [2, 0, 5]].

A transaction will be given as a tuple (x, y, z). Note that x ? y and z > 0.
Person's IDs may not be linear, e.g. we could have the persons 0, 1, 2 or we could also have the persons 0, 2, 6.

Example 1: \newline
Input:
[[0,1,10], [2,0,5]]
Output:
2
Explanation:
Person \#0 gave person \#1 \$10.
Person \#2 gave person \#0 \$5.
Two transactions are needed. One way to settle the debt is person \#1 pays person \#0 and \#2 \$5 each.

Example 2: \newline
Input:
[[0,1,10], [1,0,1], [1,2,5], [2,0,5]]
Output:
1

Explanation:
\begin{enumerate}
    \item Person \#0 gave person \#1 \$10.
    \item Person \#1 gave person \#0 \$1.
    \item Person \#1 gave person \#2 \$5.
    \item Person \#2 gave person \#0 \$5.
    \item Therefore, person \#1 only need to give person \#0 \$4, and all debt is settled.
\end{enumerate}

\subsubsection{分析}
先計算每個人的 balance。然後嘗試所有的可能性,記低最細的交易次數。
Reference \myurl{https://www.youtube.com/watch?v=I8lLGTgb9LU}


\subsubsection{代碼 - DFS}
\begin{Code}
// LeetCode, Optimal Account Balancing
// dfs,時間複雜度O(),空間複雜度O()
class Solution {
public:
    int minTransfers(const vector<vector<int>>& transactions)
    {
        unordered_map<int, int> map; // key: person value: balance
        for (const auto& tran : transactions)
        {
            int personGive = tran[0];
            int personGet = tran[1];
            int balance = tran[2];

            auto giveIT = map.find(personGive);
            if (giveIT == map.end()) giveIT = map.insert(giveIT, make_pair(personGive, 0));
            auto getIT = map.find(personGet);
            if (getIT == map.end()) getIT = map.insert(getIT, make_pair(personGet, 0));

            giveIT->second += balance;
            getIT->second -= balance;
        }

        vector<int> vec; vec.reserve(map.size());
        for (const auto& m : map)
        {
            vec.push_back(m.second);
        }

        return dfs(0, vec);
    }
private:
    int dfs(size_t k, vector<int>& vec)
    {
        if (k == vec.size()) return 0;
        int cur = vec[k];
        if (cur == 0)
            return dfs(k + 1, vec);

        int minTrans = INT_MAX;
        for (size_t i = k + 1; i < vec.size(); i++)
        {
            int next = vec[i];
            if (cur * next < 0)
            {
                vec[i] = cur + next;
                minTrans = min(minTrans, 1 + dfs(k + 1, vec));
                vec[i] = next;
            }

            if (cur + next == 0) break;
        }

        return minTrans;
    }
};
\end{Code}


\subsubsection{相關題目}
\begindot
\item 無
\myenddot

\section{小結} %%%%%%%%%%%%%%%%%%%%%%%%%%%%%%
\label{sec:dfs-template}


\subsection{適用場景}

\textbf{輸入數據}:如果是遞歸數據結構,如單鏈表,二叉樹,集合,則百分之百可以用深搜;如果是非遞歸數據結構,如一維數組,二維數組,字符串,圖,則概率小一些。

\textbf{狀態轉換圖}:樹或者圖。

\textbf{求解目標}:必須要走到最深(例如對於樹,必須要走到葉子節點)才能得到一個解,這種情況適合用深搜。


\subsection{思考的步驟}
\begin{enumerate}
\item 是求路徑條數,還是路徑本身(或動作序列)?深搜最常見的三個問題,求可行解的總數,求一個可行解,求所有可行解。
    \begin{enumerate}
	\item 如果是路徑條數,則不需要存儲路徑。
    \item 如果是求路徑本身,則要用一個數組\fn{path[]}存儲路徑。跟寬搜不同,寬搜雖然最終求的也是一條路徑,但是需要存儲擴展過程中的所有路徑,在沒找到答案之前所有路徑都不能放棄;而深搜,在搜索過程中始終只有一條路徑,因此用一個數組就足夠了。
    \end{enumerate}

\item 只要求一個解,還是要求所有解?如果只要求一個解,那找到一個就可以返回;如果要求所有解,找到了一個後,還要繼續擴展,直到遍歷完。廣搜一般只要求一個解,因而不需要考慮這個問題(廣搜當然也可以求所有解,這時需要擴展到所有葉子節點,相當於在內存中存儲整個狀態轉換圖,非常佔內存,因此廣搜不適合解這類問題)。

\item 如何表示狀態?即一個狀態需要存儲哪些些必要的數據,才能夠完整提供如何擴展到下一步狀態的所有信息。跟廣搜不同,深搜的慣用寫法,不是把數據記錄在狀態\fn{struct}裏,而是添加函數參數(有時為了節省遞歸堆棧,用全局變量),\fn{struct}裏的字段與函數參數一一對應。

\item 如何擴展狀態?這一步跟上一步相關。狀態裏記錄的數據不同,擴展方法就不同。對於固定不變的數據結構(一般題目直接給出,作為輸入數據),如二叉樹,圖等,擴展方法很簡單,直接往下一層走,對於隱式圖,要先在第1步裏想清楚狀態所帶的數據,想清楚了這點,那如何擴展就很簡單了。

\item 終止條件是什麼?終止條件是指到了不能擴展的末端節點。對於樹,是葉子節點,對於圖或隱式圖,是出度為0的節點。

\item {收斂條件是什麼?收斂條件是指找到了一個合法解的時刻。如果是正向深搜(父狀態處理完了才進行遞歸,即父狀態不依賴子狀態,遞歸語句一定是在最後,尾遞歸),則是指是否達到目標狀態;如果是逆向深搜(處理父狀態時需要先知道子狀態的結果,此時遞歸語句不在最後),則是指是否到達初始狀態。

由於很多時候終止條件和收斂條件是是合二為一的,因此很多人不區分這兩種條件。仔細區分這兩種條件,還是很有必要的。

為了判斷是否到了收斂條件,要在函數接口裏用一個參數記錄當前的位置(或距離目標還有多遠)。如果是求一個解,直接返回這個解;如果是求所有解,要在這裏收集解,即把第一步中表示路徑的數組\fn{path[]}複製到解集合裏。}

\item 關於判重
    \begin{enumerate}
    \item 是否需要判重?如果狀態轉換圖是一棵樹,則不需要判重,因為在遍歷過程中不可能重複;如果狀態轉換圖是一個DAG,則需要判重。這一點跟BFS不一樣,BFS的狀態轉換圖總是DAG,必須要判重。
    \item 怎樣判重?跟廣搜相同,見第 \S \ref{sec:bfs-template} 節。同時,DAG説明存在重疊子問題,此時可以用緩存加速,見第8步。
    \end{enumerate}

\item 如何加速?
    \begin{enumerate}
    \item 剪枝。深搜一定要好好考慮怎麼剪枝,成本小收益大,加幾行代碼,就能大大加速。這裏沒有通用方法,只能具體問題具體分析,要充分觀察,充分利用各種信息來剪枝,在中間節點提前返回。
    \item 緩存。
        \begin{enumerate}
            \item 前提條件:狀態轉換圖是一個DAG。DAG=>存在重疊子問題=>子問題的解會被重複利用,用緩存自然會有加速效果。如果依賴關係是樹狀的(例如樹,單鏈表等),沒必要加緩存,因為子問題只會一層層往下,用一次就再也不會用到,加了緩存也沒什麼加速效果。
            \item 具體實現:可以用數組或HashMap。維度簡單的,用數組;維度複雜的,用HashMap,C++有\fn{map},C++ 11以後有\fn{unordered_map},比\fn{map}快。
        \end{enumerate}
    
    \end{enumerate}
\end{enumerate}

拿到一個題目,當感覺它適合用深搜解決時,在心裏面把上面8個問題默默回答一遍,代碼基本上就能寫出來了。對於樹,不需要回答第5和第8個問題。如果讀者對上面的經驗總結看不懂或感覺“不實用”,很正常,因為這些經驗總結是我做了很多題目後總結出來的,從思維的發展過程看,“經驗總結”要晚於感性認識,所以這時候建議讀者先做做前面的題目,積累一定的感性認識後,再回過頭來看這一節的總結,一定會有共鳴。


\subsection{代碼模板}

\begin{Codex}[label=dfs_template.cpp]
/**
 * dfs模板.
 * @param[in] input 輸入數據指針
 * @param[out] path 當前路徑,也是中間結果
 * @param[out] result 存放最終結果
 * @param[inout] cur or gap 標記當前位置或距離目標的距離
 * @return 路徑長度,如果是求路徑本身,則不需要返回長度
 */
void dfs(type &input, type &path, type &result, int cur or gap) {
    if (數據非法) return 0;   // 終止條件
    if (cur == input.size()) { // 收斂條件
    // if (gap == 0) {
        將path放入result
    }

    if (可以剪枝) return;

    for(...) { // 執行所有可能的擴展動作
        執行動作,修改path
        dfs(input, step + 1 or gap--, result);
        恢復path
    }
}
\end{Codex}


\subsection{深搜與回溯法的區別}
深搜(Depth-first search, DFS)的定義見\myurl{http://en.wikipedia.org/wiki/Depth_first_search},回溯法(backtracking)的定義見\myurl{http://en.wikipedia.org/wiki/Backtracking}

\textbf{回溯法 = 深搜 + 剪枝}。一般大家用深搜時,或多或少會剪枝,因此深搜與回溯法沒有什麼不同,可以在它們之間畫上一個等號。本書同時使用深搜和回溯法兩個術語,但讀者可以認為二者等價。

深搜一般用遞歸(recursion)來實現,這樣比較簡潔。

深搜能夠在候選答案生成到一半時,就進行判斷,拋棄不滿足要求的答案,所以深搜比暴力搜索法要快。


\subsection{深搜與遞歸的區別}
\label{sec:dfs-vs-recursion}

深搜經常用遞歸(recursion)來實現,二者常常同時出現,導致很多人誤以為他倆是一個東西。

深搜,是邏輯意義上的算法,遞歸,是一種物理意義上的實現,它和迭代(iteration)是對應的。深搜,可以用遞歸來實現,也可以用棧來實現;而遞歸,一般總是用來實現深搜。可以説,\textbf{遞歸一定是深搜,深搜不一定用遞歸}。

遞歸有兩種加速策略,一種是\textbf{剪枝(prunning)},對中間結果進行判斷,提前返回;一種是\textbf{緩存},緩存中間結果,防止重複計算,用空間換時間。

其實,遞歸+緩存,就是 memoization。所謂\textbf{memoization}(翻譯為備忘錄法,見第 \S \ref{sec:dp-vs-memoization}節),就是"top-down with cache"(自頂向下+緩存),它是Donald Michie 在1968年創造的術語,表示一種優化技術,在top-down 形式的程序中,使用緩存來避免重複計算,從而達到加速的目的。

\textbf{memoization 不一定用遞歸},就像深搜不一定用遞歸一樣,可以在迭代(iterative)中使用 memoization 。\textbf{遞歸也不一定用 memoization},可以用memoization來加速,但不是必須的。只有當遞歸使用了緩存,它才是 memoization 。

既然遞歸一定是深搜,為什麼很多書籍都同時使用這兩個術語呢?在遞歸味道更濃的地方,一般用遞歸這個術語,在深搜更濃的場景下,用深搜這個術語,讀者心裏要弄清楚他倆大部分時候是一回事。在單鏈表、二叉樹等遞歸數據結構上,遞歸的味道更濃,這時用遞歸這個術語;在圖、隱式圖等數據結構上,深搜的味道更濃,這時用深搜這個術語。
