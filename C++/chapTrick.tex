\chapter{編程技巧}

在判斷兩個浮點數a和b是否相等時,不要用\fn{a==b},應該判斷二者之差的絕對值\fn{fabs(a-b)}是否小於某個閾值,例如\fn{1e-9}。

判斷一個整數是否是為奇數,用\fn{x \% 2 != 0},不要用\fn{x \% 2 == 1},因為x可能是負數。

用\fn{char}的值作為數組下標(例如,統計字符串中每個字符出現的次數),要考慮到\fn{char}可能是負數。有的人考慮到了,先強制轉型為\fn{unsigned int}再用作下標,這仍然是錯的。正確的做法是,先強制轉型為\fn{unsigned char},再用作下標。這涉及C++整型提升的規則,就不詳述了。

以下是關於STL使用技巧的,很多條款來自《Effective STL》這本書。

\subsubsection{vector和string優先於動態分配的數組}

首先,在性能上,由於\fn{vector}能夠保證連續內存,因此一旦分配了後,它的性能跟原始數組相當;

其次,如果用new,意味着你要確保後面進行了delete,一旦忘記了,就會出現BUG,且這樣需要都寫一行delete,代碼不夠短;

再次,聲明多維數組的話,只能一個一個new,例如:
\begin{Code}
int** ary = new int*[row_num];
for(int i = 0; i < row_num; ++i)
    ary[i] = new int[col_num];
\end{Code}
用vector的話一行代碼搞定,
\begin{Code}
vector<vector<int> > ary(row_num, vector<int>(col_num, 0));
\end{Code}

\subsubsection{使用reserve來避免不必要的重新分配}
