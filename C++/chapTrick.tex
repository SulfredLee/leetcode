\chapter{編程技巧}

在判斷兩個浮點數a和b是否相等時,不要用\fn{a==b},應該判斷二者之差的絕對值\fn{fabs(a-b)}是否小於某個閾值,例如\fn{1e-9}。

判斷一個整數是否是為奇數,用\fn{x \% 2 != 0},不要用\fn{x \% 2 == 1},因為x可能是負數。

用\fn{char}的值作為數組下標(例如,統計字符串中每個字符出現的次數),要考慮到\fn{char}可能是負數。有的人考慮到了,先強制轉型為\fn{unsigned int}再用作下標,這仍然是錯的。正確的做法是,先強制轉型為\fn{unsigned char},再用作下標。這涉及C++整型提升的規則,就不詳述了。

以下是關於STL使用技巧的,很多條款來自《Effective STL》這本書。

\subsubsection{vector和string優先於動態分配的數組}

首先,在性能上,由於\fn{vector}能夠保證連續內存,因此一旦分配了後,它的性能跟原始數組相當;

其次,如果用new,意味着你要確保後面進行了delete,一旦忘記了,就會出現BUG,且這樣需要都寫一行delete,代碼不夠短;

再次,聲明多維數組的話,只能一個一個new,例如:
\begin{Code}
int** ary = new int*[row_num];
for(int i = 0; i < row_num; ++i)
    ary[i] = new int[col_num];
\end{Code}
用vector的話一行代碼搞定,
\begin{Code}
vector<vector<int> > ary(row_num, vector<int>(col_num, 0));
\end{Code}

\subsubsection{使用reserve來避免不必要的重新分配}
\subsubsection{Things about binary search}
The difference between using index or length.

\begin{Code}
// LeetCode, 
// 時間複雜度O(logn),空間複雜度O(1)
class solution{
public:
    int findIndex(const vector<int>& vec, int target) {
        int left, right;
        left = 0; right = (int)vec.size() - 1; // right is using index
        
        while (left <= right) { // this line is different
            int mid = left + (right - left) / 2;
            if (vec[mid] < target)
                left = ++mid;
            else if (vec[mid] > target)
                right = --mid; // this line is different
            else
                return mid;
        }
        return -1;
    }
};
\end{Code}

\begin{Code}
// LeetCode, 
// 時間複雜度O(logn),空間複雜度O(1)
class solution{
public:
    int findIndex(const vector<int>& vec, int target) {
        int left, right;
        left = 0; right = (int)vec.size(); // right is using length 
        
        while (left < right) { // this line is different
            int mid = left + (right - left) / 2;
            if (vec[mid] < target)
                left = ++mid;
            else if (vec[mid] > target)
                right = mid; // this line is different
            else
                return mid;
        }
        return -1;
    }
};
\end{Code}

\subsubsection{Hash map in c++ with custom class}

\begin{Code}
using h = std::hash<int>;
auto hash = [](const Node& n)
              {return ((17 * 31 + h()(n.a)) * 31 + h()(n.b)) * 31 + h()(n.c);};
auto equal = [](const Node& l, const Node& r)
               {return l.a == r.a && l.b == r.b && l.c == r.c;};
std::unordered_map<Node, int, decltype(hash), decltype(equal)> m(8, hash, equal);
\end{Code}

\begin{Code}
namespace std {
template<>
struct hash<pair<Iterator, Iterator>> {
    size_t operator()(pair<Iterator, Iterator> const& p) const {
        return ((size_t) &(*p.first)) ^ ((size_t) &(*p.second));
    }
};
}
\end{Code}

\subsubsection{Priority queue in c++ with custom class}

\begin{Code}
auto Compare = [](const auto& first, const auto& second)
                  { return first.second > second.second; };
priority_queue<pair<int, int>, vector<pair<int, int>>, decltype(Compare)>
                  pq(Compare);
\end{Code}

\subsubsection{Use unordered_map with pair}

\begin{Code}
namespace std
{
    template <>
        struct hash<pair<int, int>>
        {
            size_t operator()(pair<int, int> const& p) const
            {
                return ((size_t)&(p.first) ^ (size_t)&(p.second));
            }
        };
}

class Solution {
private:
    unordered_set<pair<int, int>> m_example;
}
\end{Code}