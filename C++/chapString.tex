\chapter{字符串}


\section{Valid Palindrome} %%%%%%%%%%%%%%%%%%%%%%%%%%%%%%
\label{sec:valid-palindrome}


\subsubsection{描述}
Given a string, determine if it is a palindrome, considering only alphanumeric characters and ignoring cases.

For example,\\
\code{"A man, a plan, a canal: Panama"} is a palindrome.\\
\code{"race a car"} is not a palindrome.

Note:
Have you consider that the string might be empty? This is a good question to ask during an interview.

For the purpose of this problem, we define empty string as valid palindrome.


\subsubsection{分析}
無


\subsubsection{代碼}
\begin{Code}
// Leet Code, Valid Palindrome
// 時間複雜度O(n),空間複雜度O(1)
class Solution {
public:
    bool isPalindrome(string s) {
        transform(s.begin(), s.end(), s.begin(), ::tolower);
        auto left = s.begin(), right = prev(s.end());
        while (left < right) {
            if (!::isalnum(*left))  ++left;
            else if (!::isalnum(*right)) --right;
            else if (*left != *right) return false;
            else { left++, right--; }
        }
        return true;
    }
};
\end{Code}


\subsubsection{相關題目}
\begindot
\item Palindrome Number, 見 \S \ref{sec:palindrome-number}
\myenddot


\section{Implement strStr()} %%%%%%%%%%%%%%%%%%%%%%%%%%%%%%
\label{sec:strstr}


\subsubsection{描述}
Implement strStr().

Returns a pointer to the first occurrence of needle in haystack, or null if needle is not part of haystack.


\subsubsection{分析}
暴力算法的複雜度是 $O(m*n)$,代碼如下。更高效的的算法有KMP算法、Boyer-Mooer算法和Rabin-Karp算法。面試中暴力算法足夠了,一定要寫得沒有BUG。


\subsubsection{暴力匹配}
\begin{Code}
// LeetCode, Implement strStr()
// 暴力解法,時間複雜度O(N*M),空間複雜度O(1)
class Solution {
public:
    int strStr(const string& haystack, const string& needle) {
        if (needle.empty()) return 0;

        const int N = haystack.size() - needle.size() + 1;
        for (int i = 0; i < N; i++) {
            int j = i;
            int k = 0;
            while (j < haystack.size() && k < needle.size() && haystack[j] == needle[k]) {
                j++;
                k++;
            }
            if (k == needle.size()) return i;
        }
        return -1;
    }
};
\end{Code}


\subsubsection{KMP}
\begin{Code}
// LeetCode, Implement strStr()
// KMP,時間複雜度O(N+M),空間複雜度O(M)
class Solution {
public:
    int strStr(const string& haystack, const string& needle) {
        return kmp(haystack.c_str(), needle.c_str());
    }
private:
    /*
     * @brief 計算部分匹配表,即next數組.
     *
     * @param[in] pattern 模式串
     * @param[out] next next數組
     * @return 無
     */
    static void compute_prefix(const char *pattern, int next[]) {
        int i;
        int j = -1;
        const int m = strlen(pattern);

        next[0] = j;
        for (i = 1; i < m; i++) {
            while (j > -1 && pattern[j + 1] != pattern[i]) j = next[j];

            if (pattern[i] == pattern[j + 1]) j++;
            next[i] = j;
        }
    }

    /*
     * @brief KMP算法.
     *
     * @param[in] text 文本
     * @param[in] pattern 模式串
     * @return 成功則返回第一次匹配的位置,失敗則返回-1
     */
    static int kmp(const char *text, const char *pattern) {
        int i;
        int j = -1;
        const int n = strlen(text);
        const int m = strlen(pattern);
        if (n == 0 && m == 0) return 0; /* "","" */
        if (m == 0) return 0;  /* "a","" */
        int *next = (int*)malloc(sizeof(int) * m);

        compute_prefix(pattern, next);

        for (i = 0; i < n; i++) {
            while (j > -1 && pattern[j + 1] != text[i]) j = next[j];

            if (text[i] == pattern[j + 1]) j++;
            if (j == m - 1) {
                free(next);
                return i-j;
            }
        }

        free(next);
        return -1;
    }
};
\end{Code}


\subsubsection{相關題目}
\begindot
\item String to Integer (atoi) ,見 \S \ref{sec:string-to-integer}
\myenddot


\section{String to Integer (atoi)} %%%%%%%%%%%%%%%%%%%%%%%%%%%%%%
\label{sec:string-to-integer}


\subsubsection{描述}
Implement \fn{atoi} to convert a string to an integer.

\textbf{Hint}: Carefully consider all possible input cases. If you want a challenge, please do not see below and ask yourself what are the possible input cases.

\textbf{Notes}: It is intended for this problem to be specified vaguely (ie, no given input specs). You are responsible to gather all the input requirements up front.

\textbf{Requirements for atoi}:

The function first discards as many whitespace characters as necessary until the first non-whitespace character is found. Then, starting from this character, takes an optional initial plus or minus sign followed by as many numerical digits as possible, and interprets them as a numerical value.

The string can contain additional characters after those that form the integral number, which are ignored and have no effect on the behavior of this function.

If the first sequence of non-whitespace characters in str is not a valid integral number, or if no such sequence exists because either str is empty or it contains only whitespace characters, no conversion is performed.

If no valid conversion could be performed, a zero value is returned. If the correct value is out of the range of representable values, \code{INT_MAX (2147483647)} or \code{INT_MIN (-2147483648)} is returned.

\subsubsection{分析}
細節題。注意幾個測試用例:
\begin{enumerate}
\item 不規則輸入,但是有效,"-3924x8fc", "  +  413",
\item 無效格式," ++c", " ++1"
\item 溢出數據,"2147483648"
\end{enumerate}

\subsubsection{代碼}
\begin{Code}
// LeetCode, String to Integer (atoi)
// 時間複雜度O(n),空間複雜度O(1)
class Solution {
public:
    int myAtoi(const string &str) {
        int num = 0;
        int sign = 1;
        const int n = str.length();
        int i = 0;

        while (str[i] == ' ' && i < n) i++;
        if (i == n) return 0;

        if (str[i] == '+') {
            i++;
        } else if (str[i] == '-') {
            sign = -1;
            i++;
        }

        for (; i < n; i++) {
            if (str[i] < '0' || str[i] > '9')
                break;
            if (num > INT_MAX / 10 ||
                            (num == INT_MAX / 10 &&
                                    (str[i] - '0') > INT_MAX % 10)) {
                return sign == -1 ? INT_MIN : INT_MAX;
            }
            num = num * 10 + str[i] - '0';
        }
        return num * sign;
    }
};
\end{Code}


\subsubsection{相關題目}
\begindot
\item Implement strStr() ,見 \S \ref{sec:strstr}
\myenddot


\section{Add Binary} %%%%%%%%%%%%%%%%%%%%%%%%%%%%%%
\label{sec:add-binary}


\subsubsection{描述}
Given two binary strings, return their sum (also a binary string).

For example,
\begin{Code}
a = "11"
b = "1"
\end{Code}
Return {\small \fontspec{Latin Modern Mono} "100"}.


\subsubsection{分析}
無


\subsubsection{代碼}
\begin{Code}
//LeetCode, Add Binary
// 時間複雜度O(n),空間複雜度O(1)
class Solution {
public:
    string addBinary(string a, string b) {
        string result;
        const size_t n = a.size() > b.size() ? a.size() : b.size();
        reverse(a.begin(), a.end());
        reverse(b.begin(), b.end());
        int carry = 0;
        for (size_t i = 0; i < n; i++) {
            const int ai = i < a.size() ? a[i] - '0' : 0;
            const int bi = i < b.size() ? b[i] - '0' : 0;
            const int val = (ai + bi + carry) % 2;
            carry = (ai + bi + carry) / 2;
            result.insert(result.begin(), val + '0');
        }
        if (carry == 1) {
            result.insert(result.begin(), '1');
        }
        return result;
    }
};
\end{Code}


\subsubsection{相關題目}
\begindot
\item Add Two Numbers, 見 \S \ref{sec:add-two-numbers}
\myenddot


\section{Longest Palindromic Substring} %%%%%%%%%%%%%%%%%%%%%%%%%%%%%%
\label{sec:longest-palindromic-substring}


\subsubsection{描述}
Given a string $S$, find the longest palindromic substring in $S$. You may assume that the maximum length of $S$ is 1000, and there exists one unique longest palindromic substring.


\subsubsection{分析}
最長迴文子串,非常經典的題。

思路一:暴力枚舉,以每個元素為中間元素,同時從左右出發,複雜度$O(n^2)$。

思路二:記憶化搜索,複雜度$O(n^2)$。設\fn{f[i][j]} 表示[i,j]之間的最長迴文子串,遞推方程如下:
\begin{Code}
f[i][j] = if (i == j) S[i]
          if (S[i] == S[j] && f[i+1][j-1] == S[i+1][j-1]) S[i][j]
          else max(f[i+1][j-1], f[i][j-1], f[i+1][j])
\end{Code}

思路三:動規,複雜度$O(n^2)$。設狀態為\fn{f(i,j)},表示區間[i,j]是否為迴文串,則狀態轉移方程為
$$
f(i,j)=\begin{cases}
true & ,i=j\\
S[i]=S[j] & , j = i + 1 \\
S[i]=S[j] \text{ and } f(i+1, j-1) & , j > i + 1
\end{cases}
$$

思路四:Manacher’s Algorithm, 複雜度$O(n)$。詳細解釋見 \myurl{http://leetcode.com/2011/11/longest-palindromic-substring-part-ii.html} 。


\subsubsection{備忘錄法}
\begin{Code}
// LeetCode, Longest Palindromic Substring
// 備忘錄法,會超時
// 時間複雜度O(n^2),空間複雜度O(n^2)
typedef string::const_iterator Iterator;

namespace std {
template<>
struct hash<pair<Iterator, Iterator>> {
    size_t operator()(pair<Iterator, Iterator> const& p) const {
        return ((size_t) &(*p.first)) ^ ((size_t) &(*p.second));
    }
};
}

class Solution {
public:
    string longestPalindrome(string const& s) {
        cache.clear();
        return cachedLongestPalindrome(s.begin(), s.end());
    }

private:
    unordered_map<pair<Iterator, Iterator>, string> cache;

    string longestPalindrome(Iterator first, Iterator last) {
        size_t length = distance(first, last);

        if (length < 2) return string(first, last);

        auto s = cachedLongestPalindrome(next(first), prev(last));

        if (s.length() == length - 2 && *first == *prev(last))
            return string(first, last);

        auto s1 = cachedLongestPalindrome(next(first), last);
        auto s2 = cachedLongestPalindrome(first, prev(last));

        // return max(s, s1, s2)
        if (s.size() > s1.size()) return s.size() > s2.size() ? s : s2;
        else return s1.size() > s2.size() ? s1 : s2;
    }

    string cachedLongestPalindrome(Iterator first, Iterator last) {
        auto key = make_pair(first, last);
        auto pos = cache.find(key);

        if (pos != cache.end()) return pos->second;
        else return cache[key] = longestPalindrome(first, last);
    }
};
\end{Code}


\subsubsection{動規}
\begin{Code}
// LeetCode, Longest Palindromic Substring
// 動規,時間複雜度O(n^2),空間複雜度O(n^2)
class Solution {
public:
    string longestPalindrome(const string& s) {
        const int n = s.size();
        bool f[n][n];
        fill_n(&f[0][0], n * n, false);
        // 用 vector 會超時
        //vector<vector<bool> > f(n, vector<bool>(n, false));
        size_t max_len = 1, start = 0;  // 最長迴文子串的長度,起點

        for (size_t i = 0; i < s.size(); i++) {
            f[i][i] = true;
            for (size_t j = 0; j < i; j++) {  // [j, i]
                f[j][i] = (s[j] == s[i] && (i - j < 2 || f[j + 1][i - 1]));
                if (f[j][i] && max_len < (i - j + 1)) {
                    max_len = i - j + 1;
                    start = j;
                }
            }
        }
        return s.substr(start, max_len);
    }
};
\end{Code}


\subsubsection{Manacher’s Algorithm}
\begin{Code}
// LeetCode, Longest Palindromic Substring
// Manacher’s Algorithm
// 時間複雜度O(n),空間複雜度O(n)
class Solution {
public:
    // Transform S into T.
    // For example, S = "abba", T = "^#a#b#b#a#$".
    // ^ and $ signs are sentinels appended to each end to avoid bounds checking
    string preProcess(const string& s) {
        int n = s.length();
        if (n == 0) return "^$";

        string ret = "^";
        for (int i = 0; i < n; i++) ret += "#" + s.substr(i, 1);

        ret += "#$";
        return ret;
    }

    string longestPalindrome(string s) {
        string T = preProcess(s);
        const int n = T.length();
        // 以T[i]為中心,向左/右擴張的長度,不包含T[i]自己,
        // 因此 P[i]是源字符串中迴文串的長度
        int P[n];
        int C = 0, R = 0;

        for (int i = 1; i < n - 1; i++) {
            int i_mirror = 2 * C - i; // equals to i' = C - (i-C)

            P[i] = (R > i) ? min(R - i, P[i_mirror]) : 0;

            // Attempt to expand palindrome centered at i
            while (T[i + 1 + P[i]] == T[i - 1 - P[i]])
                P[i]++;

            // If palindrome centered at i expand past R,
            // adjust center based on expanded palindrome.
            if (i + P[i] > R) {
                C = i;
                R = i + P[i];
            }
        }

        // Find the maximum element in P.
        int max_len = 0;
        int center_index = 0;
        for (int i = 1; i < n - 1; i++) {
            if (P[i] > max_len) {
                max_len = P[i];
                center_index = i;
            }
        }

        return s.substr((center_index - 1 - max_len) / 2, max_len);
    }
};
\end{Code}


\subsubsection{相關題目}
\begindot
\item 無
\myenddot


\section{Regular Expression Matching} %%%%%%%%%%%%%%%%%%%%%%%%%%%%%%
\label{sec:regular-expression-matching}


\subsubsection{描述}
Implement regular expression matching with support for \fn{'.'} and \fn{'*'}.

\fn{'.'} Matches any single character.
\fn{'*'} Matches zero or more of the preceding element.

The matching should cover the entire input string (not partial).

The function prototype should be:
\begin{Code}
bool isMatch(const char *s, const char *p)
\end{Code}

Some examples:
\begin{Code}
isMatch("aa","a") → false
isMatch("aa","aa") → true
isMatch("aaa","aa") → false
isMatch("aa", "a*") → true
isMatch("aa", ".*") → true
isMatch("ab", ".*") → true
isMatch("aab", "c*a*b") → true
\end{Code}


\subsubsection{分析}
這是一道很有挑戰的題。


\subsubsection{遞歸版}
\begin{Code}
// LeetCode, Regular Expression Matching
// 遞歸版,時間複雜度O(n),空間複雜度O(1)
class Solution {
public:
    bool isMatch(const string& s, const string& p) {
        return isMatch(s.c_str(), p.c_str());
    }
private:
    bool isMatch(const char *s, const char *p) {
        if (*p == '\0') return *s == '\0';

        // next char is not '*', then must match current character
        if (*(p + 1) != '*') {
            if (*p == *s || (*p == '.' && *s != '\0'))
                return isMatch(s + 1, p + 1);
            else
                return false;
        } else { // next char is '*'
            while (*p == *s || (*p == '.' && *s != '\0')) {
                if (isMatch(s, p + 2))
                    return true;
                s++;
            }
            return isMatch(s, p + 2);
        }
    }
};
\end{Code}


\subsubsection{迭代版}
\begin{Code}

\end{Code}


\subsubsection{相關題目}
\begindot
\item Wildcard Matching, 見 \S \ref{sec:wildcard-matching}
\myenddot


\section{Wildcard Matching} %%%%%%%%%%%%%%%%%%%%%%%%%%%%%%
\label{sec:wildcard-matching}


\subsubsection{描述}
Implement wildcard pattern matching with support for \fn{'?'} and \fn{'*'}.

\fn{'?'} Matches any single character.
\fn{'*'} Matches any sequence of characters (including the empty sequence).

The matching should cover the entire input string (not partial).

The function prototype should be:
\begin{Code}
bool isMatch(const char *s, const char *p)
\end{Code}

Some examples:
\begin{Code}
isMatch("aa","a") → false
isMatch("aa","aa") → true
isMatch("aaa","aa") → false
isMatch("aa", "*") → true
isMatch("aa", "a*") → true
isMatch("ab", "?*") → true
isMatch("aab", "c*a*b") → false
\end{Code}


\subsubsection{分析}
跟上一題很類似。

主要是\fn{'*'}的匹配問題。\fn{p}每遇到一個\fn{'*'},就保留住當前\fn{'*'}的座標和\fn{s}的座標,然後\fn{s}從前往後掃描,如果不成功,則\fn{s++},重新掃描。


\subsubsection{遞歸版}
\begin{Code}
// LeetCode, Wildcard Matching
// 遞歸版,會超時,用於幫助理解題意
// 時間複雜度O(n!*m!),空間複雜度O(n)
class Solution {
public:
    bool isMatch(const string& s, const string& p) {
        return isMatch(s.c_str(), p.c_str());
    }
private:
    bool isMatch(const char *s, const char *p) {
        if (*p == '*') {
            while (*p == '*') ++p;  //skip continuous '*'
            if (*p == '\0') return true;
            while (*s != '\0' && !isMatch(s, p)) ++s;

            return *s != '\0';
        }
        else if (*p == '\0' || *s == '\0') return *p == *s;
        else if (*p == *s || *p == '?') return isMatch(++s, ++p);
        else return false;
    }
};
\end{Code}


\subsubsection{迭代版}
\begin{Code}
// LeetCode, Wildcard Matching
// 迭代版,時間複雜度O(n*m),空間複雜度O(1)
class Solution {
public:
    bool isMatch(const string& s, const string& p) {
        return isMatch(s.c_str(), p.c_str());
    }
private:
    bool isMatch(const char *s, const char *p) {
        bool star = false;
        const char *str, *ptr;
        for (str = s, ptr = p; *str != '\0'; str++, ptr++) {
            switch (*ptr) {
            case '?':
                break;
            case '*':
                star = true;
                s = str, p = ptr;
                while (*p == '*') p++;  //skip continuous '*'
                if (*p == '\0') return true;
                str = s - 1;
                ptr = p - 1;
                break;
            default:
                if (*str != *ptr) {
                    // 如果前面沒有'*',則匹配不成功
                    if (!star) return false;
                    s++;
                    str = s - 1;
                    ptr = p - 1;
                }
            }
        }
        while (*ptr == '*') ptr++;
        return (*ptr == '\0');
    }
};
\end{Code}


\subsubsection{相關題目}
\begindot
\item Regular Expression Matching, 見 \S \ref{sec:regular-expression-matching}
\myenddot


\section{Longest Common Prefix} %%%%%%%%%%%%%%%%%%%%%%%%%%%%%%
\label{sec:longest-common-prefix}


\subsubsection{描述}
Write a function to find the longest common prefix string amongst an array of strings.


\subsubsection{分析}
從位置0開始,對每一個位置比較所有字符串,直到遇到一個不匹配。


\subsubsection{縱向掃描}
\begin{Code}
// LeetCode, Longest Common Prefix
// 縱向掃描,從位置0開始,對每一個位置比較所有字符串,直到遇到一個不匹配
// 時間複雜度O(n1+n2+...)
// @author 周倩 (http://weibo.com/zhouditty)
class Solution {
public:
    string longestCommonPrefix(vector<string> &strs) {
        if (strs.empty()) return "";

        for (int idx = 0; idx < strs[0].size(); ++idx) { // 縱向掃描
            for (int i = 1; i < strs.size(); ++i) {
                if (strs[i][idx] != strs[0][idx]) return strs[0].substr(0,idx);
            }
        }
        return strs[0];
    }
};
\end{Code}


\subsubsection{橫向掃描}
\begin{Code}
// LeetCode, Longest Common Prefix
// 橫向掃描,每個字符串與第0個字符串,從左到右比較,直到遇到一個不匹配,
// 然後繼續下一個字符串
// 時間複雜度O(n1+n2+...)
class Solution {
public:
    string longestCommonPrefix(vector<string> &strs) {
        if (strs.empty()) return "";

        int right_most = strs[0].size() - 1;
        for (size_t i = 1; i < strs.size(); i++)
            for (int j = 0; j <= right_most; j++)
                if (strs[i][j] != strs[0][j])  // 不會越界,請參考string::[]的文檔
                    right_most = j - 1;

        return strs[0].substr(0, right_most + 1);
    }
};
\end{Code}


\subsubsection{相關題目}
\begindot
\item 無
\myenddot


\section{Valid Number} %%%%%%%%%%%%%%%%%%%%%%%%%%%%%%
\label{sec:valid-number}


\subsubsection{描述}
Validate if a given string is numeric.

Some examples:
\begin{Code}
"0" => true
" 0.1 " => true
"abc" => false
"1 a" => false
"2e10" => true
\end{Code}

Note: It is intended for the problem statement to be ambiguous. You should gather all requirements up front before implementing one.


\subsubsection{分析}
細節實現題。

本題的功能與標準庫中的\fn{strtod()}功能類似。


\subsubsection{有限自動機}
\begin{Code}
// LeetCode, Valid Number
// @author 龔陸安 (http://weibo.com/luangong)
// finite automata,時間複雜度O(n),空間複雜度O(n)
class Solution {
public:
    bool isNumber(const string& s) {
        enum InputType {
            INVALID,    // 0
            SPACE,      // 1
            SIGN,       // 2
            DIGIT,      // 3
            DOT,        // 4
            EXPONENT,   // 5
            NUM_INPUTS  // 6
        };
        const int transitionTable[][NUM_INPUTS] = {
                -1, 0, 3, 1, 2, -1, // next states for state 0
                -1, 8, -1, 1, 4, 5,     // next states for state 1
                -1, -1, -1, 4, -1, -1,     // next states for state 2
                -1, -1, -1, 1, 2, -1,     // next states for state 3
                -1, 8, -1, 4, -1, 5,     // next states for state 4
                -1, -1, 6, 7, -1, -1,     // next states for state 5
                -1, -1, -1, 7, -1, -1,     // next states for state 6
                -1, 8, -1, 7, -1, -1,     // next states for state 7
                -1, 8, -1, -1, -1, -1,     // next states for state 8
                };

        int state = 0;
        for (auto ch : s) {
            InputType inputType = INVALID;
            if (isspace(ch))
                inputType = SPACE;
            else if (ch == '+' || ch == '-')
                inputType = SIGN;
            else if (isdigit(ch))
                inputType = DIGIT;
            else if (ch == '.')
                inputType = DOT;
            else if (ch == 'e' || ch == 'E')
                inputType = EXPONENT;

            // Get next state from current state and input symbol
            state = transitionTable[state][inputType];

            // Invalid input
            if (state == -1) return false;
        }
        // If the current state belongs to one of the accepting (final) states,
        // then the number is valid
        return state == 1 || state == 4 || state == 7 || state == 8;

    }
};
\end{Code}


\subsubsection{使用strtod()}
\begin{Code}
// LeetCode, Valid Number
// @author 連城 (http://weibo.com/lianchengzju)
// 偷懶,直接用 strtod(),時間複雜度O(n)
class Solution {
public:
    bool isNumber (const string& s) {
        return isNumber(s.c_str());
    }
private:
    bool isNumber (char const* s) {
        char* endptr;
        strtod (s, &endptr);

        if (endptr == s) return false;

        for (; *endptr; ++endptr)
            if (!isspace (*endptr)) return false;

        return true;
    }
};
\end{Code}


\subsubsection{相關題目}
\begindot
\item 無
\myenddot


\section{Integer to Roman} %%%%%%%%%%%%%%%%%%%%%%%%%%%%%%
\label{sec:integer-to-roman}


\subsubsection{描述}
Given an integer, convert it to a roman numeral.

Input is guaranteed to be within the range from 1 to 3999.


\subsubsection{分析}
無


\subsubsection{代碼}
\begin{Code}
// LeetCode, Integer to Roman
// 時間複雜度O(num),空間複雜度O(1)
class Solution {
public:
    string intToRoman(int num) {
        const int radix[] = {1000, 900, 500, 400, 100, 90,
                50, 40, 10, 9, 5, 4, 1};
        const string symbol[] = {"M", "CM", "D", "CD", "C", "XC",
                "L", "XL", "X", "IX", "V", "IV", "I"};

        string roman;
        for (size_t i = 0; num > 0; ++i) {
            int count = num / radix[i];
            num %= radix[i];
            for (; count > 0; --count) roman += symbol[i];
        }
        return roman;
    }
};
\end{Code}


\subsubsection{相關題目}
\begindot
\item Roman to Integer, 見 \S \ref{sec:roman-to-integer}
\myenddot


\section{Roman to Integer} %%%%%%%%%%%%%%%%%%%%%%%%%%%%%%
\label{sec:roman-to-integer}


\subsubsection{描述}
Given a roman numeral, convert it to an integer.

Input is guaranteed to be within the range from 1 to 3999.


\subsubsection{分析}
從前往後掃描,用一個臨時變量記錄分段數字。

如果當前比前一個大,説明這一段的值應該是當前這個值減去上一個值。比如\fn{IV = 5 – 1};否則,將當前值加入到結果中,然後開始下一段記錄。比如\fn{VI = 5 + 1, II=1+1}


\subsubsection{代碼}
\begin{Code}
// LeetCode, Roman to Integer
// 時間複雜度O(n),空間複雜度O(1)
class Solution {
public:
    inline int map(const char c) {
        switch (c) {
        case 'I': return 1;
        case 'V': return 5;
        case 'X': return 10;
        case 'L': return 50;
        case 'C': return 100;
        case 'D': return 500;
        case 'M': return 1000;
        default: return 0;
        }
    }

    int romanToInt(const string& s) {
        int result = 0;
        for (size_t i = 0; i < s.size(); i++) {
            if (i > 0 && map(s[i]) > map(s[i - 1])) {
                result += (map(s[i]) - 2 * map(s[i - 1]));
            } else {
                result += map(s[i]);
            }
        }
        return result;
    }
};
\end{Code}


\subsubsection{相關題目}
\begindot
\item Integer to Roman, 見 \S \ref{sec:integer-to-roman}
\myenddot


\section{Count and Say} %%%%%%%%%%%%%%%%%%%%%%%%%%%%%%
\label{sec:count-and-say}


\subsubsection{描述}
The count-and-say sequence is the sequence of integers beginning as follows:
\begin{Code}
1, 11, 21, 1211, 111221, ...
\end{Code}

\fn{1} is read off as \fn{"one 1"} or \fn{11}.

\fn{11} is read off as \fn{"two 1s"} or \fn{21}.

\fn{21} is read off as \fn{"one 2"}, then \fn{"one 1"} or \fn{1211}.

Given an integer $n$, generate the nth sequence.

Note: The sequence of integers will be represented as a string.


\subsubsection{分析}
模擬。


\subsubsection{代碼}
\begin{Code}
// LeetCode, Count and Say
// @author 連城 (http://weibo.com/lianchengzju)
// 時間複雜度O(n^2),空間複雜度O(n)
class Solution {
public:
    string countAndSay(int n) {
        string s("1");

        while (--n)
            s = getNext(s);

        return s;
    }

    string getNext(const string &s) {
        stringstream ss;

        for (auto i = s.begin(); i != s.end(); ) {
            auto j = find_if(i, s.end(), bind1st(not_equal_to<char>(), *i));
            ss << distance(i, j) << *i;
            i = j;
        }

        return ss.str();
    }
};
\end{Code}


\subsubsection{相關題目}
\begindot
\item 無
\myenddot


\section{Anagrams} %%%%%%%%%%%%%%%%%%%%%%%%%%%%%%
\label{sec:anagrams}


\subsubsection{描述}
Given an array of strings, return all groups of strings that are anagrams.

Note: All inputs will be in lower-case.


\subsubsection{分析}
Anagram(迴文構詞法)是指打亂字母順序從而得到新的單詞,比如 \fn{"dormitory"} 打亂字母順序會變成 \fn{"dirty room"} ,\fn{"tea"} 會變成\fn{"eat"}。

迴文構詞法有一個特點:單詞裏的字母的種類和數目沒有改變,只是改變了字母的排列順序。因此,將幾個單詞按照字母順序排序後,若它們相等,則它們屬於同一組 anagrams 。


\subsubsection{代碼}
\begin{Code}
// LeetCode, Anagrams
// 時間複雜度O(n),空間複雜度O(n)
class Solution {
public:
    vector<string> anagrams(vector<string> &strs) {
        unordered_map<string, vector<string> > group;
        for (const auto &s : strs) {
            string key = s;
            sort(key.begin(), key.end());
            group[key].push_back(s);
        }

        vector<string> result;
        for (auto it = group.cbegin(); it != group.cend(); it++) {
            if (it->second.size() > 1)
                result.insert(result.end(), it->second.begin(), it->second.end());
        }
        return result;
    }
};
\end{Code}


\subsubsection{相關題目}
\begindot
\item 無
\myenddot


\section{Simplify Path} %%%%%%%%%%%%%%%%%%%%%%%%%%%%%%
\label{sec:simplify-path}


\subsubsection{描述}
Given an absolute path for a file (Unix-style), simplify it.

For example, \\
path = \fn{"/home/"}, => \fn{"/home"} \\
path = \fn{"/a/./b/../../c/"}, => \fn{"/c"} \\

Corner Cases:
\begindot
\item Did you consider the case where path = \fn{"/../"}? 
In this case, you should return \fn{"/"}.
\item 
Another corner case is the path might contain multiple slashes \fn{'/'} together, such as \fn{"/home//foo/"}.
In this case, you should ignore redundant slashes and return \fn{"/home/foo"}.
\myenddot


\subsubsection{分析}
很有實際價值的題目。


\subsubsection{代碼}
\begin{Code}
// LeetCode, Simplify Path
// 時間複雜度O(n),空間複雜度O(n)
class Solution {
public:
    string simplifyPath(const string& path) {
        vector<string> dirs; // 當做棧

        for (auto i = path.begin(); i != path.end();) {
            ++i;

            auto j = find(i, path.end(), '/');
            auto dir = string(i, j);

            if (!dir.empty() && dir != ".") {// 當有連續 '///'時,dir 為空
                if (dir == "..") {
                    if (!dirs.empty())
                        dirs.pop_back();
                } else
                    dirs.push_back(dir);
            }

            i = j;
        }

        stringstream out;
        if (dirs.empty()) {
            out << "/";
        } else {
            for (auto dir : dirs)
                out << '/' << dir;
        }

        return out.str();
    }
};
\end{Code}


\subsubsection{相關題目}
\begindot
\item 無
\myenddot


\section{Length of Last Word} %%%%%%%%%%%%%%%%%%%%%%%%%%%%%%
\label{sec:length-of-last-word}


\subsubsection{描述}
Given a string s consists of upper/lower-case alphabets and empty space characters \fn{' '}, return the length of last word in the string.

If the last word does not exist, return 0.

Note: A word is defined as a character sequence consists of non-space characters only.

For example, 
Given \fn{s = "Hello World"},
return 5.


\subsubsection{分析}
細節實現題。


\subsubsection{用 STL}
\begin{Code}
// LeetCode, Length of Last Word
// 偷懶,用 STL
// 時間複雜度O(n),空間複雜度O(1)
class Solution {
public:
    int lengthOfLastWord(const string& s) {
        auto first = find_if(s.rbegin(), s.rend(), ::isalpha);
        auto last = find_if_not(first, s.rend(), ::isalpha);
        return distance(first, last);
    }
};
\end{Code}


\subsubsection{順序掃描}
\begin{Code}
// LeetCode, Length of Last Word
// 順序掃描,記錄每個 word 的長度
// 時間複雜度O(n),空間複雜度O(1)
class Solution {
public:
    int lengthOfLastWord(const string& s) {
        return lengthOfLastWord(s.c_str());
    }
private:
    int lengthOfLastWord(const char *s) {
        int len = 0;
        while (*s) {
            if (*s++ != ' ')
                ++len;
            else if (*s && *s != ' ')
                len = 0;
        }
        return len;
    }
};
\end{Code}


\subsubsection{相關題目}
\begindot
\item 無
\myenddot

\section{Longest Happy Prefix} %%%%%%%%%%%%%%%%%%%%%%%%%%%%%%
\label{sec:longest-happy-prefix}


\subsubsection{描述}
A string is called a happy prefix if is a non-empty prefix which is also a suffix (excluding itself).
Given a string s. Return the longest happy prefix of s .
Return an empty string if no such prefix exists.

\begin{Code}
Example 1:
Input: s = "level"
Output: "l"
Explanation: s contains 4 prefix excluding itself ("l", "le", "lev", "leve")
, and suffix ("l", "el", "vel", "evel").
The largest prefix which is also suffix is given by "l".

Example 2:
Input: s = "ababab"
Output: "abab"
Explanation: "abab" is the largest prefix which is also suffix.
They can overlap in the original string.

Example 3:
Input: s = "leetcodeleet"
Output: "leet"

Example 4:
Input: s = "a"
Output: ""
\end{Code}

Constraints:
1 $<=$ s.length $<=$ $10^5$
s contains only lowercase English letters.

\subsubsection{分析}
細節實現題。


\subsubsection{順序掃描}
\begin{Code}
// LeetCode, Longest happy prefix
// try all possible match
// 時間複雜度O(n^2),空間複雜度O(1)
class Solution {
public:
    string longestPrefix(string s) {
        int N = (int)s.length();
        if (N < 2) return "";

        for (int i = 1; i < N; i++) {
            int j = i;
            int k = 0;
            while (k < N && j < N && s[k++] == s[j++]);

            if (j == N) return s.substr(i, N - i);
        }

        return "";
    }
};
\end{Code}

\subsubsection{Rolling Hash}
\begin{Code}
// LeetCode, Longest happy prefix
// rolling hashing
// 時間複雜度O(n),空間複雜度O(1)
class Solution {
public:
    string longestPrefix(string s) {
        int times = 2;
        int64_t prefixHash = 0;
        int64_t suffixHash = 0;
        int64_t multiplier = 1;
        int64_t len = 0;
        int64_t mod = 1000000007;//use some large prime as a modulo to avoid overflow errors, e.g. 10 ^ 9 + 7.
        int N = s.length();
        for (int i = 0; i < N - 1; i++) {
            prefixHash = (prefixHash * times + s[i]) % mod;
            suffixHash = (multiplier * s[N - 1 - i] + suffixHash) % mod;
            if (prefixHash == suffixHash)
                len = i + 1;
            multiplier = multiplier * times % mod;
        }
        return s.substr(0, len);
    }
};
\end{Code}

\subsubsection{相關題目}
\begindot
\item 無
\myenddot

\section{Decode String} %%%%%%%%%%%%%%%%%%%%%%%%%%%%%%
\label{sec:decode-string}


\subsubsection{描述}
Given an encoded string, return its decoded string.

The encoding rule is: k[encoded_string], where the encoded_string inside the square brackets is being repeated exactly k times. Note that k is guaranteed to be a positive integer.

You may assume that the input string is always valid; No extra white spaces, square brackets are well-formed, etc.

Furthermore, you may assume that the original data does not contain any digits and that digits are only for those repeat numbers, k. For example, there won't be input like 3a or 2[4].


\subsubsection{分析}
Nil

\subsubsection{遞歸版}
\begin{Code}
// LeetCode
// 時間複雜度O(),空間複雜度O()
class Solution {
public:
    string decodeString(string s) {
        return decodeString(s.begin(), s.end());
    }
private:
    template <class BidirecIT>
        string decodeString(BidirecIT first, BidirecIT end)
    {
        if (first == end) return "";

        string result;
        while (first != end)
        {
            if (isdigit(*first))
            {
                // 取得所有數字
                auto cur = first;
                while (cur != end) { if (*cur == '[') break; cur++; }
                int num = stol(string(first, cur));
                first = cur;

                // 取得接下來的字串
                auto first2 = next(first);
                auto end2 = first2;
                int curLeft = 1; // 現在 [ 的數目
                while (end2 != end && curLeft > 0)
                {
                    if (*end2 == '[')
                        curLeft++;
                    else if (*end2 == ']')
                        curLeft--;
                    end2++;
                }
                string result2 = decodeString(first2, prev(end2));
                string result3;
                // 重覆地放入
                for (int i = 0; i < num; i++)
                    result3 += result2;
                result += result3;

                first = end2;
            }
            else
            {
                // 取得所有字母
                auto cur = first;
                while (cur != end) { if (*cur >= '0' && *cur <= '9') break; cur++; }
                result += string(first, cur);
                first = cur;
            }
        }
        return result;
    }
};
\end{Code}


\subsubsection{相關題目}
\begindot
\item 無
\myenddot
