\chapter{Database}
Database questions
\newline

\section{Combine Two Tables} %%%%%%%%%%%%%%%%%%%%%%%%%%%%%%
\label{sec:combine-two-tables}


\subsubsection{描述}
SQL Schema

Table: Person
\begin{Code}
+-------------+---------+
| Column Name | Type    |
+-------------+---------+
| PersonId    | int     |
| FirstName   | varchar |
| LastName    | varchar |
+-------------+---------+
PersonId is the primary key column for this table.
\end{Code}

Table: Address
\begin{Code}
+-------------+---------+
| Column Name | Type    |
+-------------+---------+
| AddressId   | int     |
| PersonId    | int     |
| City        | varchar |
| State       | varchar |
+-------------+---------+
AddressId is the primary key column for this table.
\end{Code}

Write a SQL query for a report that provides the following information for each person in the Person table, regardless if there is an address for each of those people:

\subsubsection{outer join}
\begin{Code}
  // Note: Using where clause to filter the records will fail if there is
  // no address information for a person because it will not display the name information.
SELECT
    FirstName, LastName, City, State
FROM
    Person LEFT JOIN Address
ON
    Person.PersonID = Address.PersonID
;
\end{Code}

\section{Second Highest Salary} %%%%%%%%%%%%%%%%%%%%%%%%%%%%%%
\label{sec:second-highest-salary}


\subsubsection{描述}
SQL Schema

Table: Employee
\begin{Code}
+----+--------+
| Id | Salary |
+----+--------+
| 1  | 100    |
| 2  | 200    |
| 3  | 300    |
+----+--------+
\end{Code}

For example, given the above Employee table, the query should return 200 as the second highest salary. If there is no second highest salary, then the query should return null.
\begin{Code}
+---------------------+
| SecondHighestSalary |
+---------------------+
| 200                 |
+---------------------+
\end{Code}

Write a SQL query to get the second highest salary from the Employee table.

\subsubsection{sub-query and LIMIT clause}
\begin{Code}
  // Note: Use a sub-query to handle only one record in this table
SELECT
    (SELECT DISTINCT
        Salary
     FROM
         Employee
     ORDER BY Salary DESC
     LIMIT 1 OFFSET 1) AS SecondHighestSalary
;
\end{Code}

\subsubsection{IFNULL and LIMIT}
\begin{Code}
  // Note: Use a sub-query to handle only one record in this table
SELECT
    IFNULL(
      (SELECT DISTINCT
           Salary
       FROM
           Employee
       ORDER BY Salary DESC
       LIMIT 1 OFFSET 1),
    NULL) AS SecondHighestSalary
;
\end{Code}

\section{Nth Highest Salary} %%%%%%%%%%%%%%%%%%%%%%%%%%%%%%
\label{sec:nth-highest-salary}


\subsubsection{描述}
SQL Schema

Table: Employee
\begin{Code}
+----+--------+
| Id | Salary |
+----+--------+
| 1  | 100    |
| 2  | 200    |
| 3  | 300    |
+----+--------+
\end{Code}

For example, given the above Employee table, the nth highest salary where n = 2 is 200. If there is no nth highest salary, then the query should return null.
\begin{Code}
+------------------------+
| getNthHighestSalary(2) |
+------------------------+
| 200                    |
+------------------------+
\end{Code}

Write a SQL query to get the nth highest salary from the Employee table.

\subsubsection{IFNULL and LIMIT}
\begin{Code}
CREATE FUNCTION getNthHighestSalary(N INT) RETURNS INT
BEGIN

SET N=N-1;
  RETURN (
      # Write your MySQL query statement below.
      select
          ifnull(
              (select distinct
                   salary
               from
                   employee
               order by  salary desc
               limit N,1),
          null)
  );
END
\end{Code}

\section{Rank Scores} %%%%%%%%%%%%%%%%%%%%%%%%%%%%%%
\label{sec:rank-scores}


\subsubsection{描述}
SQL Schema

\begin{Code}
Create table If Not Exists Scores (Id int, Score DECIMAL(3,2))
Truncate table Scores
insert into Scores (Id, Score) values ('1', '3.5')
insert into Scores (Id, Score) values ('2', '3.65')
insert into Scores (Id, Score) values ('3', '4.0')
insert into Scores (Id, Score) values ('4', '3.85')
insert into Scores (Id, Score) values ('5', '4.0')
insert into Scores (Id, Score) values ('6', '3.65')
\end{Code}

Write a SQL query to rank scores. If there is a tie between two scores, both should have the same ranking. Note that after a tie, the next ranking number should be the next consecutive integer value. In other words, there should be no "holes" between ranks.

Table: Scores
\begin{Code}
+----+-------+
| Id | Score |
+----+-------+
| 1  | 3.50  |
| 2  | 3.65  |
| 3  | 4.00  |
| 4  | 3.85  |
| 5  | 4.00  |
| 6  | 3.65  |
+----+-------+
\end{Code}

For example, given the above Scores table, your query should generate the following report (order by highest score):
\begin{Code}
+-------+---------+
| score | Rank    |
+-------+---------+
| 4.00  | 1       |
| 4.00  | 1       |
| 3.85  | 2       |
| 3.65  | 3       |
| 3.65  | 3       |
| 3.50  | 4       |
+-------+---------+
\end{Code}

\textbf{Important Note}: For MySQL solutions, to escape reserved words used as column names, you can use an apostrophe before and after the keyword. For example `Rank`.

\subsubsection{Dense_Rank}
\begin{Code}
SELECT Score, dense_rank() OVER(ORDER BY Score DESC) AS `Rank` FROM Scores;
\end{Code}

\section{Consecutive Numbers} %%%%%%%%%%%%%%%%%%%%%%%%%%%%%%
\label{sec:consecutive-numbers}


\subsubsection{描述}
SQL Schema

\begin{Code}
Create table If Not Exists Logs (Id int, Num int)
Truncate table Logs
insert into Logs (Id, Num) values ('1', '1')
insert into Logs (Id, Num) values ('2', '1')
insert into Logs (Id, Num) values ('3', '1')
insert into Logs (Id, Num) values ('4', '2')
insert into Logs (Id, Num) values ('5', '1')
insert into Logs (Id, Num) values ('6', '2')
insert into Logs (Id, Num) values ('7', '2')
\end{Code}

Write an SQL query to find all numbers that appear at least three times consecutively.
Return the result table in any order.
The query result format is in the following example:

Table: Logs
\begin{Code}
+-------------+---------+
| Column Name | Type    |
+-------------+---------+
| id          | int     |
| num         | varchar |
+-------------+---------+
id is the primary key for this table.
\end{Code}


\begin{Code}
Logs table:
+----+-----+
| Id | Num |
+----+-----+
| 1  | 1   |
| 2  | 1   |
| 3  | 1   |
| 4  | 2   |
| 5  | 1   |
| 6  | 2   |
| 7  | 2   |
+----+-----+

Result table:
+-----------------+
| ConsecutiveNums |
+-----------------+
| 1               |
+-----------------+
1 is the only number that appears consecutively for at least three times.
\end{Code}


\subsubsection{multiple table}
\begin{Code}
SELECT DISTINCT
    l1.Num AS ConsecutiveNums
FROM
    Logs l1,
    Logs l2,
    Logs l3
WHERE
    l1.Id = l2.Id - 1
    AND l2.Id = l3.Id - 1
    AND l1.Num = l2.Num
    AND l2.Num = l3.Num
;
\end{Code}

\section{Employees Earning More Than Their Managers} %%%%%%%%%%%%%%%%%%%%%%%%%%%%%%
\label{sec:employees-earning-more-than-their-managers}


\subsubsection{描述}
SQL Schema

\begin{Code}
Create table If Not Exists Employee (Id int, Name varchar(255), Salary int, ManagerId int)
Truncate table Employee
insert into Employee (Id, Name, Salary, ManagerId) values ('1', 'Joe', '70000', '3')
insert into Employee (Id, Name, Salary, ManagerId) values ('2', 'Henry', '80000', '4')
insert into Employee (Id, Name, Salary, ManagerId) values ('3', 'Sam', '60000', 'None')
insert into Employee (Id, Name, Salary, ManagerId) values ('4', 'Max', '90000', 'None')
\end{Code}

The Employee table holds all employees including their managers. Every employee has an Id, and there is also a column for the manager Id.

Table: Employee
\begin{Code}
+----+-------+--------+-----------+
| Id | Name  | Salary | ManagerId |
+----+-------+--------+-----------+
| 1  | Joe   | 70000  | 3         |
| 2  | Henry | 80000  | 4         |
| 3  | Sam   | 60000  | NULL      |
| 4  | Max   | 90000  | NULL      |
+----+-------+--------+-----------+
\end{Code}

Given the Employee table, write a SQL query that finds out employees who earn more than their managers. For the above table, Joe is the only employee who earns more than his manager.

\begin{Code}
+----------+
| Employee |
+----------+
| Joe      |
+----------+
\end{Code}


\subsubsection{multiple table}
\begin{Code}
SELECT
    a.Name AS 'Employee'
FROM
    Employee AS a,
    Employee AS b
WHERE
    a.ManagerId = b.Id
        AND a.Salary > b.Salary
;
\end{Code}

\subsubsection{join}
\begin{Code}
SELECT
     a.NAME AS Employee
FROM Employee AS a JOIN Employee AS b
     ON a.ManagerId = b.Id
     AND a.Salary > b.Salary
;
\end{Code}

\section{Duplicate Emails} %%%%%%%%%%%%%%%%%%%%%%%%%%%%%%
\label{sec:duplicate-emails}


\subsubsection{描述}
SQL Schema

\begin{Code}
Create table If Not Exists Person (Id int, Email varchar(255))
Truncate table Person
insert into Person (Id, Email) values ('1', 'a@b.com')
insert into Person (Id, Email) values ('2', 'c@d.com')
insert into Person (Id, Email) values ('3', 'a@b.com')
\end{Code}

Write a SQL query to find all duplicate emails in a table named Person.

Table: Person
\begin{Code}
+----+---------+
| Id | Email   |
+----+---------+
| 1  | a@b.com |
| 2  | c@d.com |
| 3  | a@b.com |
+----+---------+
\end{Code}

For example, your query should return the following for the above table:

\begin{Code}
+---------+
| Email   |
+---------+
| a@b.com |
+---------+
\end{Code}


\subsubsection{Using GROUP BY and a temporary table}
\begin{Code}
select Email from
(
  select Email, count(Email) as num
  from Person
  group by Email
) as statistic
where num > 1
;
\end{Code}

\subsubsection{Using GROUP BY and HAVING condition}
\begin{Code}
select Email
from Person
group by Email
having count(Email) > 1;
\end{Code}

\section{Customers Who Never Order} %%%%%%%%%%%%%%%%%%%%%%%%%%%%%%
\label{sec:customers-who-never-order}


\subsubsection{描述}
SQL Schema

\begin{Code}
Create table If Not Exists Customers (Id int, Name varchar(255))
Create table If Not Exists Orders (Id int, CustomerId int)
Truncate table Customers
insert into Customers (Id, Name) values ('1', 'Joe')
insert into Customers (Id, Name) values ('2', 'Henry')
insert into Customers (Id, Name) values ('3', 'Sam')
insert into Customers (Id, Name) values ('4', 'Max')
Truncate table Orders
insert into Orders (Id, CustomerId) values ('1', '3')
insert into Orders (Id, CustomerId) values ('2', '1')
\end{Code}

Suppose that a website contains two tables, the Customers table and the Orders table. Write a SQL query to find all customers who never order anything.

Table: Customers.
\begin{Code}
+----+-------+
| Id | Name  |
+----+-------+
| 1  | Joe   |
| 2  | Henry |
| 3  | Sam   |
| 4  | Max   |
+----+-------+
\end{Code}

Table: Orders.

\begin{Code}
+----+------------+
| Id | CustomerId |
+----+------------+
| 1  | 3          |
| 2  | 1          |
+----+------------+
\end{Code}

Using the above tables as example, return the following:

\begin{Code}
+-----------+
| Customers |
+-----------+
| Henry     |
| Max       |
+-----------+
\end{Code}

\subsubsection{Not In}
\begin{Code}
select customers.name as 'Customers'
from customers
where customers.id not in
(
    select customerid from orders
);
\end{Code}

\subsubsection{Left Join}
\begin{Code}
SELECT Name AS 'Customers'
FROM Customers c
LEFT JOIN Orders o
ON c.Id = o.CustomerId
WHERE o.CustomerId IS NULL;
\end{Code}

\section{Department Highest Salary} %%%%%%%%%%%%%%%%%%%%%%%%%%%%%%
\label{sec:department-highest-salary}


\subsubsection{描述}
SQL Schema

\begin{Code}
Create table If Not Exists Employee (Id int, Name varchar(255), Salary int, DepartmentId int)
Create table If Not Exists Department (Id int, Name varchar(255))
Truncate table Employee
insert into Employee (Id, Name, Salary, DepartmentId) values ('1', 'Joe', '70000', '1')
insert into Employee (Id, Name, Salary, DepartmentId) values ('2', 'Jim', '90000', '1')
insert into Employee (Id, Name, Salary, DepartmentId) values ('3', 'Henry', '80000', '2')
insert into Employee (Id, Name, Salary, DepartmentId) values ('4', 'Sam', '60000', '2')
insert into Employee (Id, Name, Salary, DepartmentId) values ('5', 'Max', '90000', '1')
Truncate table Department
insert into Department (Id, Name) values ('1', 'IT')
insert into Department (Id, Name) values ('2', 'Sales')
\end{Code}

The Employee table holds all employees. Every employee has an Id, a salary, and there is also a column for the department Id.

\begin{Code}
+----+-------+--------+--------------+
| Id | Name  | Salary | DepartmentId |
+----+-------+--------+--------------+
| 1  | Joe   | 70000  | 1            |
| 2  | Jim   | 90000  | 1            |
| 3  | Henry | 80000  | 2            |
| 4  | Sam   | 60000  | 2            |
| 5  | Max   | 90000  | 1            |
+----+-------+--------+--------------+
\end{Code}

The Department table holds all departments of the company.
\begin{Code}
+----+----------+
| Id | Name     |
+----+----------+
| 1  | IT       |
| 2  | Sales    |
+----+----------+
\end{Code}

Write a SQL query to find employees who have the highest salary in each of the departments. For the above tables, your SQL query should return the following rows (order of rows does not matter).
\begin{Code}
+------------+----------+--------+
| Department | Employee | Salary |
+------------+----------+--------+
| IT         | Max      | 90000  |
| IT         | Jim      | 90000  |
| Sales      | Henry    | 80000  |
+------------+----------+--------+
\end{Code}

\subsubsection{Join, In}
\begin{Code}
# This example shows that IN can match multiple fields together
SELECT
    Department.name AS 'Department',
    Employee.name AS 'Employee',
    Salary
FROM
    Employee
        JOIN
    Department ON Employee.DepartmentId = Department.Id
WHERE
    (Employee.DepartmentId , Salary) IN
    (   SELECT
            DepartmentId, MAX(Salary)
        FROM
            Employee
        GROUP BY DepartmentId
    )
;
\end{Code}

\section{Department Top Three Salaries} %%%%%%%%%%%%%%%%%%%%%%%%%%%%%%
\label{sec:department-top-three-salaries}


\subsubsection{描述}
SQL Schema

\begin{Code}
Create table If Not Exists Employee (Id int, Name varchar(255), Salary int, DepartmentId int)
Create table If Not Exists Department (Id int, Name varchar(255))
Truncate table Employee
insert into Employee (Id, Name, Salary, DepartmentId) values ('1', 'Joe', '85000', '1')
insert into Employee (Id, Name, Salary, DepartmentId) values ('2', 'Henry', '80000', '2')
insert into Employee (Id, Name, Salary, DepartmentId) values ('3', 'Sam', '60000', '2')
insert into Employee (Id, Name, Salary, DepartmentId) values ('4', 'Max', '90000', '1')
insert into Employee (Id, Name, Salary, DepartmentId) values ('5', 'Janet', '69000', '1')
insert into Employee (Id, Name, Salary, DepartmentId) values ('6', 'Randy', '85000', '1')
insert into Employee (Id, Name, Salary, DepartmentId) values ('7', 'Will', '70000', '1')
Truncate table Department
insert into Department (Id, Name) values ('1', 'IT')
insert into Department (Id, Name) values ('2', 'Sales')
\end{Code}

Table: Employee

\begin{Code}
+--------------+---------+
| Column Name  | Type    |
+--------------+---------+
| Id           | int     |
| Name         | varchar |
| Salary       | int     |
| DepartmentId | int     |
+--------------+---------+
Id is the primary key for this table.
Each row contains the ID, name, salary, and department of one employee.
\end{Code}

Table: Department
\begin{Code}
+-------------+---------+
| Column Name | Type    |
+-------------+---------+
| Id          | int     |
| Name        | varchar |
+-------------+---------+
Id is the primary key for this table.
Each row contains the ID and the name of one department.
\end{Code}

A company's executives are interested in seeing who earns the most money in each of the company's departments. A high earner in a department is an employee who has a salary in the top three unique salaries for that department.

Write an SQL query to find the employees who are high earners in each of the departments.

Return the result table in any order.

The query result format is in the following example:
\begin{Code}
Employee table:
+----+-------+--------+--------------+
| Id | Name  | Salary | DepartmentId |
+----+-------+--------+--------------+
| 1  | Joe   | 85000  | 1            |
| 2  | Henry | 80000  | 2            |
| 3  | Sam   | 60000  | 2            |
| 4  | Max   | 90000  | 1            |
| 5  | Janet | 69000  | 1            |
| 6  | Randy | 85000  | 1            |
| 7  | Will  | 70000  | 1            |
+----+-------+--------+--------------+

Department table:
+----+-------+
| Id | Name  |
+----+-------+
| 1  | IT    |
| 2  | Sales |
+----+-------+

Result table:
+------------+----------+--------+
| Department | Employee | Salary |
+------------+----------+--------+
| IT         | Max      | 90000  |
| IT         | Joe      | 85000  |
| IT         | Randy    | 85000  |
| IT         | Will     | 70000  |
| Sales      | Henry    | 80000  |
| Sales      | Sam      | 60000  |
+------------+----------+--------+

In the IT department:
- Max earns the highest unique salary
- Both Randy and Joe earn the second-highest unique salary
- Will earns the third-highest unique salary

In the Sales department:
- Henry earns the highest salary
- Sam earns the second-highest salary
- There is no third-highest salary as there are only two employees
\end{Code}

\subsubsection{Join, sub-query}
\begin{Code}
SELECT
    d.Name AS 'Department', e1.Name AS 'Employee', e1.Salary
FROM
    Employee e1
        JOIN
    Department d ON e1.DepartmentId = d.Id
WHERE
    3 > (SELECT
            COUNT(DISTINCT e2.Salary)
        FROM
            Employee e2
        WHERE
            e2.Salary > e1.Salary
                AND e1.DepartmentId = e2.DepartmentId
        )
;
\end{Code}

\subsubsection{Join, dense_rank}
\begin{Code}
SELECT d.NAME  AS Department,
       a. NAME AS Employee,
       a. salary
FROM   (SELECT e.*,
               Dense_rank()
                 OVER (
                   partition BY departmentid
                   ORDER BY salary DESC) AS DeptPayRank
        FROM   employee e) a
       JOIN department d
         ON a. departmentid = d. id
WHERE  deptpayrank <= 3; 
\end{Code}